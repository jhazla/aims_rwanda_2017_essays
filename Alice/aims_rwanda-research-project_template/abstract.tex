% Abstracts are usually written in English, with a version in your
% mother tongue underneath
\chapter*{Abstract} 
%\addcontentsline{toc}{chapter}{Abstract}
% Don't change anything above this.

Reports are key source of information for all activities within  organizations. Electronic reports are generated day to day in
\Jnote{an}
unstructured way.
It is still a big challenge to know automatically what the reports are talking about.
For big organizations like International Federation of Red Cross(IFRC)
\Jnote{Space before parenthesis.}
where they work
\Jnote{s/where they work/which work}
in humanitarian domain, some information from their reports are
quietly
\Jnote{Quietly? This must be a typo and I'm not sure what you mean here.}
necessary and very important.
Within million reports, automation of extracting information saves time and
increase
\Jnote{s/increase/increases}
quality.
Needed information from the reports are called Named entities.
\Jnote{Rephrase: ``In this thesis we are concerned with extracting specific
  pieces of information called named entities, such as \textless
  example(s)\textgreater'' (or similar).
  Then mention packages you used
  as below. Do not capitalize ``named entities''.}
This research, We
\Jnote{Do not capitalize mid-sentence.}
used machine learning algorithms such as
Stanford NER, polyglot and Natural language toolkit
\Jnote{Capitalize packages names as in their documentation.}
to extract named entities from IFRC reports.
We are looking for the answer of " Who did what when How"
\Jnote{No space after opening quote.}
\Jnote{Rephrase quote to makes sense, e.g., ``Who did what, when and how?''}
from the documents. 

