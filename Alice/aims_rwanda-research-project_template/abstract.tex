% Abstracts are usually written in English, with a version in your
% mother tongue underneath
\chapter*{Abstract} 
%\addcontentsline{toc}{chapter}{Abstract}
% Don't change anything above this.
Reports are key source of information for all activities within  organizations. Electronic reports are generated day to day in an 
unstructured way.
It is still a big challenge to know automatically what the reports are talking about.
For big organizations like International Federation of Red Cross (IFRC) which work
%\Jnote{an}
%\Jnote{Space before parenthesis.}
%\Jnote{s/where they work/which work}
%quietly
%\Jnote{Quietly? This must be a typo and I'm not sure what you mean here.}
%necessary and 
%They generated million reports, 
%\Jnote{s/increase/increases}
%\Jnote{Capitalize packages names as in their documentation.}
in humanitarian domain, some information from their reports are very important.
Automation of extracting information saves time and
increases quality. In this research, we are concerned with extracting specific pieces of information called named entities, such as names of persons who participated in IFRC activities, locations, organizations, budget, etc.
%\Jnote{Rephrase: ``In this thesis we are concerned with extracting specific pieces of information called named entities, such as \textless example(s)\textgreater'' (or similar).Then mention packages you used as below. Do not capitalize ``named entities''.}
We used machine learning algorithms such as
Stanford NER, Polyglot and Natural Language ToolKit
to extract named entities from IFRC reports.
We were looking for the answer of "Who did what, when and how?" from the reports.
%\Jnote{No space after opening quote.}

%\Jnote{Rephrase quote to makes sense, e.g., }

