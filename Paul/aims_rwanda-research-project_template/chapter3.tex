\chapter{Methodology}
\section{Unsupervised Learning (SL)}
This is a machine learning task of inferring a function to describe hidden structure from unlabelled data. This type of ML does not require any prior manual categorization of observations in the data. The distinction between supervised learning and unsupervised learning (UL) is that in supervised learning there is evaluation of accuracy of the algorithm used used, because data fed to the learner is unlabelled . Also one advantage of UL over SL is that time and cost is saved in labelling as required in SL. 
\section{Natural Language Processing (NLP)}
It is a multidisciplinary area that deals with the automatic processing of human language. 
This automation allows communication between humans and computers. The computer accept input in the form of text or speech and then produces structured representations showing the meaning of those strings as their output.
\section{Vector Space}

%Theorems before the chapter's first section will be dot-zero, 
%and their numbering is completely wrong. You can avoid this
%by simply always starting a chapter with a section. Ta Da! 
%It will probably help you structure your essay anyway. 
%
%\begin{thm}[My Theorem2]
%This is my theorem2.
%\end{thm}
%\begin{proof}
%And it has no proof2.
%\end{proof}
%
%\section{See?}
%
%Text text text text text text text text text text text text text text
%text text text text text text text text text text text text text text
%text text text text text text text text text text text text text text
%text text text text text text text text text text text text text text
%text text text text text.
%
%\begin{thm}[My Theorem2]
%This is my theorem2.
%\end{thm}
%\begin{proof}
%And it has no proof2.
%\end{proof}
%
%Text text text text text text text text text text text text text text
%text text text text text text text text text text text text text text
%text text text text text text text text text text text text text text
%text text text text text text text text text text text text text text
%text text text text text.
%
%\begin{align} % do not use eqnarray. 
%\label{2ya}
%x & = y + y\\
%\label{2yb}
%& = 2y
%\end{align}
%see equations \eqref{2ya} and \ref{2yb}
%
%\section{More}
%
%Here's a conjecture
%\begin{conj}
%The washing operation has fixed points.
%\end{conj}
%
%and here's an example
%
%\begin{exa}
%100 FRW coin.
%\end{exa}
%
%\subsection{This is a subsection}
%
%\section{This is a section}
