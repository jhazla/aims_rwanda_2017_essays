
% Abstracts are usually written in English, with a version in your
% mother tongue underneath
\chapter*{Abstract} 
\addcontentsline{toc}{chapter}{Abstract}
% Don't change anything above this.
<<<<<<< HEAD
Information retrieval is one  important but challenging practice carried by organizations that deal with data. For example, we can be given thousands of documents and be interested in a single piece of information from one of them. Manually going through document by document  in search of the information can be tedious and time consuming. It would be helpful to at least narrow the search by identifying which documents are relevant to the topic our information belongs.


Topic modelling  provides a simple way to understand the contents and extract information from  large documents without  reading through them. We used the Latent Dirichlet Allocation (LDA) for this task. The LDA  is a statistical model that groups words that occur together in document based on context and then assigns a probability value for each word in the group. These groups are called the "topics" and each group represents the summary content of one or more documents. 
As a case study we used reports from the International Federation of Red Cross and Crescent Societies (IFRC). 
=======
Information retrieval is one  important but challenging practice carried by organizations that deals with data.
\Jnote{s/deals/deal}
For example given thousands of documents and the only interest is just a single information from these numerous documents.
\Jnote{Rephrase: ``For example, we can be given thousands of documents
  and be interested in just a single piece of information from
  one of them.''}
Manually going through document by document  in search of information can be tedious and time consuming.
\Jnote{Add sentence: ``It would be helpful to at least narrow the search
  by identifying which documents are relevant to the topic our information
  belongs to.''}
\Jnote{Make a new paragraph here.}
Topic modelling  provides a simple way to understand the contents and extracting
\Jnote{s/extracting/extract}
information from  large documents without  reading through them.
We used the latent Dirichlet allocation (LDA) for this task.
\Jnote{Capitalization}
The LDA  is a statistical model that group
\Jnote{s/group/groups}
words that occur together in document based on context and then assigns a probability value for each word in the group. These groups are called the "topics" and each group represents the summary content of one or more documents. 
As a case study we used reports from the
international federation of red cross and crescent societies (IFRC).
\Jnote{Capitalization.}
>>>>>>> 3e4e6217f7743e21f214deeef2ba0e72e632ebcc


%A short, abstracted description of your essay goes here. 
%It should be about 100 words long. But write it last.
%
%An abstract is not a summary of your essay: it's an abstraction of that. 
%It tells the readers why they should be interested in your essay but summarises all
%they need to know if they read no further.
%
%The writing style used in an abstract is like the style used in the rest of your essay: concise, clear and direct. 
%In the rest of the essay, however, you will introduce and use technical terms. In the abstract you should
%avoid them in order to make the result comprehensible to all.
%
%You may like to repeat the abstract in your mother tongue.

% At a unviersity like Stellenbosch you *must* produce an abstract in Afrikaans for your masters.
% At AIMS you are encouraged to repeat the abstract in your mother tongue
% French, Igbo, Mlagasy, etc. just write it using LaTeX's special
% characters.
% Arabic students see the arabic.tex file for an example
% Amharic use openoffice and export from there and import a figure here.
% Where the words do not exist put the English work in italics, or use mathematical symbols.




