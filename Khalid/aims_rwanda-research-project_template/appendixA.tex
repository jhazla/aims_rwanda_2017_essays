\chapter*{Appendix}

The codes that are used in this essay are available on \url{https://github.com/KhalidOmer/MonteCarloTutorial}. 

The file \verb+README.md+ contains a brief description of the codes and what they do as well as how to run them.

Note that all codes are written in \verb+python+, except for chapter six the codes are in \verb!C++!.

The codes are:

Chapter two:
\begin{itemize}
\item[•] Obtaining a sample with $p(x) = \frac{1}{x+\epsilon}$, the file name is \verb+accept-reject.py+.
\item[•] For the calculation of $\pi$ and the error calculation see \verb+Calculation_of_pi.y+ 

and \verb+uncertainty_in_pi_calculation.py+.
\end{itemize}  
Chapter three:
\begin{itemize}
\item[•]The two dimensions parton shower: \verb+two_D_partonshower.py+. 
\item[•]The three dimensions parton shower: \verb+partonshower3d.py+ 
\end{itemize}
Chapter four:
\begin{itemize}
\item[•] Anti-$k_t$ algorithm: \verb+antikT_algorithm.py+.
\end{itemize}
Chapter five:

In this chapter the codes are divided into two files,  one file for running the code and the other is for producing the histogram.   
\begin{itemize}
\item[•] The Pseudo-mass observable:

\verb+Pseudomass_observable.py+ 

and \verb+hist_pseudomass.py+.
\item[•] The number of constituents in the jet with highest energy observable: 

\verb+number_of_constituents_observable.py+ 

and \verb+hist_of_n_of_constituents.py+.
\item[•] The number of jets observable:

\verb+number_of_Jets_observable.py+

\verb+hist_n_of_Jets.py+.
\end{itemize}
Chapter six:

The codes are available in the folder named PythiaSimulatorAnalysis. A file called \verb+README.md+ inside this folder contains the instruction on how to run these codes.  