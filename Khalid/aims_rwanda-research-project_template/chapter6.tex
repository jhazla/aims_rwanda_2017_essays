\chapter{Conclusion}

We have implemented a toy Monte Carlo simulation for the process that happens after the collision of two protons that produces two partons (parton shower) in \verb+Python+. The products were grouped using an implementation of anti-$k_t$ algorithm and jet observables were extracted from the data. Then the we did the same simulation using \verb!C++! packages \verb+Pythia+, \verb+FastJet+ and \verb+Root+ where the \verb+Pythia+ is used for generating the parton shower and \verb+FastJet+ for analysing the data and \verb+ROOT+ for drawing the histograms. It is noted that even though unlike in \verb+Pythia+ the simple toy simulation of the parton shower in \verb+Python+ does not account for some physical properties like the particle type, the results from extracting the observables were qualitatively close in both cases.      