\chapter{On the Hales-Jewett theorem}

In this part, some notions about Hales-Jewett theorem are presented. Firstly, we will start by some basic notions on arithmetic progression, which will be important for understanding the next point. After, we will introduce some elementary notions about  Van der Warden’s theorem and  Szemerédi's theorem. We will highlight that  Van der Waerden's theorem is a particular case of Szemerédi's theorem. Ultimately, we will present the two forms of Hales-Jewett theorem and link these one to the two first theorems.

\section{Arithmetic progression}

\begin{defn}
Let be a sequence of numbers: $a_1, a_2, \cdots, a_n, \cdots.$ 

This sequence of numbers form an \textbf{arithmetic progression} if every term of this sequence is obtained by adding a constant to the previous term.
\end{defn}
The arithmetic progression is also known as an arithmetic sequence. The constant is also the difference between consecutive terms.

If $a_1$ and $a_n$ represent the first and the $n-th$ term of a sequence, and $d$ the constant, then the general term $a_n$ of this sequence is expressed as:

$$a_n=a_1+(n-1)d.$$

Knowing $a_m$ and the constant $d$, then $a_n$ can be expressed as:
 
$$a_n=a_m+(n-m)d.$$

\subsection{Arithmetic progression of length k}


Let $a$ and $d$ be two fixed numbers.

An arithmetic progression of length k is an arithmetic progression of $k$ numbers of the form $a+nd.$ $a$ is the first term of the arithmetic progression, $d$ is the difference between two consecutive terms and $n=0,1, \cdots, k-1$, that is $k$ consecutive values of $n.$ 

We denote by AP(k) or AP-k, the arithmetic progression of length $k.$


\section{Van der Waerden's theorem}

Before stating the Van der Waerden's theorem, let us introduce and define some concepts and notation.

A \textit{partition} of a set $A$ is a collection of nonempty and mutually disjoint subsets $A_i$ of $A$, such that  $A=\cup A_i$ and $A_i \cap A_j=\emptyset, \quad i\neq j.$ Thus, a partition is also a sequence $A_1, A_2, \cdots, A_n$ of mutually nonempty and  disjoint subsets of set $A$ \citep{dransfield2004}. $A_i$ are known as \textit{blocks}.

We denote by $\mathbb{Z}^+$, the set of positive integers.
Let $m \in \mathit{Z}^+$, we designate by $[m]$ the set $\{1,2, \cdots, m\}.$

Let $X$ be a set and $r$ be a positive integer. We want to colour elements of set $X$ with some colours. If $C$ represents the set of colours, then $|C|=r$ is the number of colours.

\begin{defn} An $r-colouring$ of $X$ is a mapping $c \ : \ X \longrightarrow [r].$  \label{rcol}\end{defn}

If $|X|=n$, then the number of possibilities of colouring the $n$ elements is $n^r.$

Let $Y$ be a subset of $X.$ $Y$ is $monochromatic$ when the restriction $c|Y$ is constant, that is   if $c(y)$ is the same for every $y \in Y.$

According to \cite{Polymath2009new} , the Van der Waerden is stated as follows:

\begin{thm}[Van der Waerden]
For every pair $(k,r) \in \mathbb{Z}^+ \times \mathbb{Z}^+$, there exists $N \in \mathbb{Z}^+$ such that for
every $r-colouring$ of [N] there is a monochromatic AP-k.  \label{vd1}
\end{thm}

We know that a $r-$colouring is a function called $c$ in definition \eqref{rcol}. So, in other words there exists $a$,  $d \ \in \mathbb{N}$ with $d\neq 0$ such that: $c(a)=c(a+d)=c(a+2d)=\cdots =c(a+(k-1)d).$

This Van der Waerden's theorem can also be formulated using partition \citep{dransfield2004} as:

\begin{thm}[Van der Waerden]
For every $k, r \in \mathbb{Z}^+$ , there exists $N \in \mathbb{Z}^+$ such that for every partition $A_1, \cdots , A_r$ of $[N]$, there is $i$, $1 \leq i \leq r$, such that block $A_i$ contains length at one arithmetic progression of  least $k$.   \label{vd2}
\end{thm}

The existence of the number $N$ for which any $r-$colouring of the integer $\{1, \cdots, N\}$ is certain to have a monochromatic subset of cardinality $k$ of which elements form an arithmetic progression was demonstrated constructively in 1927 by Bartel Leendert van der Waerden \cite{van1927beweis}.

\cite{graham1974short} gave a proof of this theorem. The book entitled "\textit{Purely Combinatorial Proofs of Van Der Waerden-Type Theorem}" written by \cite{gasarch2010purely} condenses  the proof of Van Der Waerden theorem. 

In this theorem, the difficult problem is to find the number $N$. The least such number is called \textit{Van der Waerden number} denoted as $W(k,l).$ The general expression of $W(k,l)$ is not known, but for some $k$ and $l$ there are exact values or there are some lower and upper bounds \citep{dransfield2004}.

$W(1,r), \ W(k,1)$ and $W(2,r)$ are known as \textit{trivial} Van der Waerden numbers. So, 

 $W(1,r)=1$: this is an $AP-1$.  $W(k,1)=k$: this is an $AP-k$. $W(2,r)=r+1$: this is an $AP-2.$ 
 

For instance, let us find the Van der Waerden number $W(2,3)$, that is a $2-$colouring  of the set $[W(2,3)]$ such that there is a monochromatic arithmetic progression of length $3.$

The value of $W(2,3)$ is greater than $8$ because for any $2-$colouring of $[n],\quad n\in \{3,4,5,6,7,8\}$, we can find a $2-$colouring which does not contain a monochromatic arithmetic progression of length 3.

So, when $W(2,3)=9$ we always find a monochromatic arithmetic progression of length 3 for any $2-$colouring of $[9].$ The table \eqref{van23} shows one of the possibilities of colouring $\{1,2,3,4,5,6,7,8,9\}.$  If the ninth number is {\color{blue} blue}, then{ \color{blue}3, 6, 9}  form an arithmetic progression. If the ninth number is {\color{red} red}, then {\color{red}1, 5, 9} form an arithmetic progression. Therefore, by adding a ninth number and colouring it using any of the two colors, we always create an form an arithmetic progression of length 3.

\begin{table}[h]
\begin{center}
\begin{tabular}{ccccccccc}
\hline
1 & 2 & 3 & 4  & 5 & 6 & 7 & 8 & 9 \\ \hline
\color{red}R & \color{blue}B & \color{blue}B & \color{red}R & \color{red}R & \color{blue}B & \color{blue}B & \color{red}R  & \\
\hline 
\end{tabular} 
\end{center}
\caption{A $2-$colouring of $\{1,2,\cdots, 9\}$} \label{van23}
\end{table}

The table \eqref{vdn} presents the 7 exact non-trivial Van der numbers  (when $k\geq 3$) \citep{dransfield2004}.

\begin{table}[h]

\begin{center}
\begin{tabular}{|c|c|c|c|}
\hline 
$k \setminus r$ & 2 & 3 & 4  \\ 
\hline 
3 & 9 & 27 & 76  \\ 
\hline 
4 & 35 & 293 &   \\ 
\hline 
5 & 178 &  &   \\ 
\hline 
6 & 1132 &  &   \\ 
\hline 

\end{tabular}
\end{center}
\caption{The 7 exact non-trivial values of Van der Waerden numbers.} \label{vdn}
\end{table} 

As related previously, searching the non-trivial value of $W(k,r)$ remains a difficult problem to find solution as long as the values of $k$ and $r$ increase. However, for some $k$ and $r$ there is an approximation of the lower or upper bound of $W(k,r)$ \citep{stevens1978computer, herwig2007new, beeler1979some, dransfield2004, brown2008bounds, rabung2012, kouril2008van}. The table \eqref{vdw1} summaries these known lower bounds and includes the seven Van der Waerden numbers known exactly.

\begin{table}[h]
\begin{tabular}{|c||c|c|c|c|c|}
\hline 
$\mathbf{k \setminus r}$ & \textbf{ 2 }& \textbf{3} & \textbf{4} & \textbf{5} & \textbf{6} \\ 
\hline  \hline
\textbf{3}  &	9	& 27 &	76 &	 \textgreater  170 &	  \textgreater  223 \\ \hline 
\textbf{4}	& 35 &	293  &	 \textgreater 1,048 &	 \textgreater 2,254 &	 \textgreater  9,778 \\ \hline 
\textbf{5}	& 178	& \textgreater 2,173 &	 \textgreater 17,705 &	 \textgreater 98,740 &	 \textgreater 98,748 \\ \hline 
\textbf{6}	& 1,132 &	 \textgreater 11,191 & \textgreater 91,331 &	 \textgreater 540,025 &	 \textgreater  816,981 \\ \hline 
\textbf{7}	&  \textgreater 3,703 &	 \textgreater  48,811 &	 \textgreater 420,217 &	 \textgreater 1,381,687 &	 \textgreater 7,465,909 \\ \hline 
\textbf{8}	&  \textgreater 11,495 &	 \textgreater 238,400 &	 \textgreater 2,388,317&	 \textgreater 10,743,258&	 \textgreater 57,445,718\\ \hline 
\textbf{9}	&  \textgreater 41,265 &	 \textgreater 932,745&	 \textgreater 10,898,729&	 \textgreater 79,706,009&	 \textgreater 458,062,329 \\ \hline 
\textbf{10}	 &  \textgreater 103,474&	 \textgreater 4,173,724&	 \textgreater 76,049,218&	 \textgreater 542,694,970 &	 \textgreater 2,615,305,384 \\ \hline 
\textbf{11}	&  \textgreater 193,941&	 \textgreater 18,603,731&	 \textgreater 305,513,57 &	 \textgreater 2,967,283,511 &	 \textgreater 3,004,668,671 \\ \hline 
\end{tabular} 

\caption{Some lower bounds and exact values of Van der Waerden numbers $W(k,r).$} \label{vdw1}
\end{table}

The estimation of lower and upper bounds is still an open problem. There exists some expression that bounded Van der Waerden numbers.  Researchers are still looking for closer bound or exact general expression of these numbers.
 \cite{erdos1952combinatorial}, cited by \cite{dransfield2004} established an inequality for the lower bound for $W(k,r).$
\begin{equation}
(2(k-1)2^{k-1})^{\frac{1}{2}} < W(k,r). \label{erdos}
\end{equation}

 \cite{brk1968} found a better bound when $k-1$= prime number  and for $r=2$ (colors). But these bounds still  require improvement.
 \begin{equation}
 (k-1)2^{k-1}<W(k,2). \label{berlk}
 \end{equation}

For $p=k-1$, the expression \eqref{berlk} becomes:
 \begin{equation}
 p2^{p}<W(p+1,r). \label{berlk1}
 \end{equation}

So, $W(6,2) >5 \times 2^5=160$, $W(8,2) >7 \times 2^7= 896$ and $W(12,2) >11 \times 2^11= 22528.$
\citep{dransfield2004} improve this lower bound by using propositional satisfiability solvers  for some small van der Waerden numbers for instance $W(2, 8) > 1322$. \cite{rabung2012} performs more. Thus, as related in table \eqref{vdw1}, most of the lower bounds used came from \cite{rabung2012}.

The best known upper bound of $W(k,r)$ is the expression  \eqref{gow} which came from the work of \cite{gowers2001new} on a  new proof of Szemerédi's theorem. The section \eqref{sz} will talk about this theorem. Szemerédi's theorem is the extension of Van der Waerden's theorem, that is Van der Waerden's theorem is a particular case of Szemerédi's theorem.

\begin{equation}
W(k,r) \leq 2^{2^{r^{2^{2^{k+9}}}}}     \label{gow}
\end{equation}

%\subsection{Off-diagonal van der Waerden numbers: w(k,r,n}


\section{Szemerédi's theorem} \label{sz}

Szemerédi's theorem is merely an extension of Van der Waerden's theorem in terms of \textit{density version}. Below, we show this implication.

Let us consider $A$ a nonempty subset of the set $[N]$. The density of $A$ inside $[N]$ is a positive real number $\delta=\frac{|A|}{N}$. It is clear that $0< \delta \leq 1.$ 

\begin{thm}[Szemerédi’s theorem]\citep{Polymath2009new}	For every  $k \in \mathbb{Z}^+$ and every $0< \delta \leq  1$ there exists an integer $N(k,\delta) \geq 1$ such that every subset $A \subseteq [N]$ of size $|A|\geq \delta N$ contains an arithmetic progression of length $k.$  \label{sz1} \end{thm}

As conjecture, Szemerédi's theorem was formulated by \cite{JLMS}. There are several proofs of this theorem. The cases $k=1$ and $k=2$ are trivial. \cite{roth1953certain, roth1970irregularities} proved the case $k=3.$ The case $k=4$ was proved by \cite{szemeredi1969sets} and he gave the general case \citep{szemeredi1975sets}.

Some of proofs necessitated the use of other theories external to combinatoric. Thus, the ergodic theory (\textit{theory related to dynamical system with invariant measures and chaos theory})  has been used to prove this theorem by \cite{furstenberg1977ergodic, furstenberg1982ergodic}. \cite{gowers1998fourier, gowers2001new}  used Fourier analysis and the inverse theory of additive  combinatorics. \cite{gowers2007hypergraph} used a hypergraph regularity lemma to prove this theorem. A  quantitative ergodic theory proof, version of \cite{furstenberg1982ergodic} has been presented  by \cite{tao2006quantitative} which does not involve some concepts used in the previous proofs: the axiom of choice, the use of infinite sets or measures, the use of the Fourier transform or inverse theorems from additive combinatorics.

\subsection{Szemerédi's theorem implies Van der Waerden's theorem.} \label{vsz}

\begin{proof}
Let us assume that all conditions of Szemerédi's theorem \eqref{sz1} are verified.  From Van der Waerden's theorem, let us show that $\forall k, r \in \mathbb{Z}^+, \quad \exists\  N(k,r) \in \mathbb{Z}^+ $ such that by $r-$colouring the set $\{1,2,\cdots, N\}$ (the number $N(k,r)$ depending on $k$ and $r$), we obtain at least one monochromatic arithmetic progression of length $k.$ Let us notice that we have shown \eqref{vd1} and \eqref{vd2} that $r-$colouring a set  is to partition it to $r$ blocks.

Let be a partition of $\{1, \cdots, N\}$ to $r$ blocks, that is $ \{1, \cdots, N\} =A_1 \cup A_2 \cup \cdots \cup A_r$, with $A_i \cap A_j \neq 0.$ This implies that $A_i \neq \emptyset.$ The color of the block $A_i$ is indicated by the number $i$. There are two blocks with the same colour.

Let $A_{max}$ be the set having the largest number of elements. By partitioning $\{1, \cdots, N\}$ to $r$ equal parts, we have: $A_{max}=A_i=\frac{N}{r}$, for $1\leq i \leq r.$

If $|A_i| < \frac{N}{r}$, for $1\leq i \leq r$, then $|A_1|+|A_2|+ \cdots +|A_r| < \frac{N}{r}+\cdots +\frac{N}{r}=\frac{rN}{r}=N$, that is $\displaystyle{ \sum_{i=1}^{r}|A_i|<N}$, therefore $A_i$ for $1\leq i \leq r$ does not form in this case a partition.

If $|A_i| \leq  \frac{N}{r}$, for $1\leq i \leq r-1$, then $\displaystyle{ \sum_{i=1}^{r-1}|A_i|\leq  \frac{(r-1)N}{r}}.$ There exists a positive integer $a$ such that  $\displaystyle{ \sum_{i=1}^{r-1}|A_i|=\frac{(r-1)N}{r}}-a.$

Thus, \begin{align*}
|A_1|+|A_2|+ \cdots +|A_r|=N & \Longleftrightarrow \frac{(r-1)N}{r}-a+|A_r|=N \\
&  \Longleftrightarrow |A_r|= \frac{N}{r}+a
\end{align*} 
Therefore, $|A_r| \geq \frac{N}{r}.$ In this case, as  $A_{max}=A_r$, then $A_{max}\geq \frac{N}{r}.$

We know that $A_{max}$ is the largest block. Let us assume that $A_{max}=A_r.$ It is clear that $|A_{max}| \geq |A_i|$, for $1\leq i \leq r-1.$

So, \begin{align*}
|A_1|+|A_2|+ \cdots +|A_r|=N & \Longleftrightarrow |A_{max}|+|A_{max}|+ \cdots +|A_{max}| \geq N \\  & \Longleftrightarrow r|A_{max}| \geq N \\
& \Longleftrightarrow |A_{max}| \geq \frac{N}{r}.
\end{align*}

$|A_{max}| \geq \frac{N}{r} \Longleftrightarrow |A_{max}| \geq \frac{1}{r} N \Longleftrightarrow |A_{max}| \geq \delta N$, with $\delta = \frac{1}{r}$, implies according to Szemerédi's theorem \eqref{sz1} $A_{max}$  contains an arithmetic progression of length $k.$

Therefore, $A_{max}$ is monochromatic arithmetic progression of length $k.$
\end{proof}

\subsection{Quantitative bounds of Szemerédi's theorem}

In the previous section \eqref{vsz} we have shown that Van der Waerden's theorem is a particular case of Szemerédi's theorem. This implies that the Szemerédi's number $N(k, \delta)$ is equal to the Van der Waerden's number $W(k,r)$ when $\delta=\frac{1}{r}.$ There is still no a general exact expression of $W(k,r),$ but there are exact values of $W(k,r)$ (7 exact values are known) for some smaller $k$ and $r$, as far as for the remain cases there are some  approximations of the lower and upper bounds of it. 

Likewise, for Szemerédi's theorem, the general exact value of $N(k, \delta)$ is not known. The seek of this number is an open problem. However, there are some quantitative approximations of the lower and upper bounds of the Szemerédi's number.  

Before giving quantitative bounds of Szemerédi's theorem existing in the literature review, let us formulate differently the Szmerédi's number. Knowing the number $N(k, \delta)$, all subset $A$ of $[N]$ such that $|A|\geq \delta N$ contains an arithmetic progression of length $k.$ Otherwise, we can define the Szemerédi's number as the largest subset of $[N]$ without containing an arithmetic progression of length $k.$ Let us denote by $r_k(N)$ the size of this largest subset. 

\begin{description}
\item[Lower bound]
\cite{behrend1946sets} proved that for $k=3$, $\epsilon >0$, $C>0$ an unspecified constant and $\log =\log_2 $:
\begin{equation}
r_3(N) \geq \frac{CN}{2^{2\sqrt{2}(1+\epsilon) \sqrt{\log N}}} \label{beh}
\end{equation}

\cite{elkin2010improved} improved the result of Behrend \eqref{beh} by a factor $\Theta (\sqrt{\log N)}$ and showed that:
\begin{equation}
r_3(N) \geq \frac{CN (\log N)^{1/4}}{2^{2\sqrt{2} \sqrt{\log N}}} \label{elkin}
\end{equation}

For $k \geq 1+2^{n-1}$,  $n=[\log k]$, $\epsilon >0$, \cite{rankin1961xxiv}, cited by \cite{o2011sets} proved that if $N$ is sufficiently large then:
\begin{equation}
r_k(N) \geq \frac{CN}{2^{n2^{(n-1)/2}(1+\epsilon) \sqrt[n]{\log N}}} \label{rak}
\end{equation}

Basing on \eqref{beh}, \eqref{elkin} and \eqref{rak}, \cite{o2011sets} constructed  the following expressions:
\begin{align}
r_3(N) \geq N \left( \frac{\sqrt{360}}{\mathrm{e}\pi^{3/2}}-\epsilon \right) \frac{\sqrt[4]{2\log N}}{4^{\sqrt{2\log N}}} \label{r31}\\ 
 \nonumber \\
r_k(N) \geq NC_k 2^{-n2^{(n-1)/2} \sqrt[n]{\log N} +\frac{1}{2n} \log \log N } \label{ob}
\end{align}

where $C_k >0$ is an unspecified constant. The expression \eqref{ob} is presently the best known  lower bounds for all $k.$

\item[Upper bound]
\cite{gowers2001new} worked on a new proof of Szemerédi's theorem and presented in this work that the upper bound of $r_k(N)$ is:
\begin{equation}
r_k(N) \leq N \left(\log \log N\right)^{-2^{-2^{k+9}}} \label{gow}
\end{equation}

where $\delta= \left(\log \log N\right)^{-2^{-2^{k+9}}}.$

\cite{bloom2016quantitative} improved the upper bound for $r_3(N):$
\begin{equation}
r_3(N) \leq C \frac{(\log \log N)^4}{\log N} N. \label{r32}
\end{equation}

For $k=4$, \cite{green2006new} improved the result \eqref{gow} of \cite{gowers2001new} as follows:
\begin{equation}
r_4(N) \leq C N\mathrm{e}^{-c\sqrt{\log \log N}} \label{r4}
\end{equation}
for some absolute constant $c > 0.$
\end{description}

Therefore, quantitative bounds of $r_k(N)$ are:
\begin{align}
NC_k 2^{-n2^{(n-1)/2} \sqrt[n]{\log N} +\frac{1}{2n} \log \log N } \leq r_k(N) \leq N\left(\log \log N\right)^{-2^{-2^{k+9}}}
\end{align}

Quantitative bounds for  $k=3$  and $k=4$ have been enhanced. Thus, for $k=4$ we have the equation \eqref{r4}. By combining  \eqref{r31} and \eqref{r32}, we have the quantitative bounds of $r_3(N)$, expressed in \eqref{r33}
\begin{equation}
N \left( \frac{\sqrt{360}}{\mathrm{e}\pi^{3/2}}-\epsilon \right) \frac{\sqrt[4]{2\log N}}{4^{\sqrt{2\log N}}} \leq r_3(N) \leq C \frac{(\log \log N)^4}{\log N} N  \label{r33}
\end{equation}

\section{Hales-Jewett theorem}

Before stating the Hales-Jewett theorem, let us introduce and define notions about combinatorial lines.  Combinatorial line is for Hales-Jewett theorem what arithmetic progression is for Van der Waerden's theorem, that is Hales-Jewett theorem is based on structures called combinatorial lines.

Let $k$ and $n$  be two positive integers. We know that $[k]^n= \underbrace{[k] \times [k] \times \cdots \times [k]}_{n \text{ set-factors  of}\  [k]}=\{(x_1,x_2,\cdots, x_n):\ x_i \in [k] \}.$ The set $[k]^n$ contains $k^n$ elements.

For instance, $k=3$ and $n=6$, an element of the set $[3]^6$ is : $121132.$

Let us consider the set $([k]\times \{x\})^n.$ Similarly, the set $([k]\times \{x\})^n$ contains $(k+1)^n$ elements.

Elements of $([k]\times \{x\})^n$ are called \textit{coordinates.} $x$ is called \textit{wildcard.}

%Given ${k,N\in{\mathbb N}}$, a ${N}-$dimensional \textit{variable word} on ${k}$ letters is an element of  $\displaystyle \left([k]\cup\{x\}\right)^d\setminus[k]^N.$ Similarly, given ${k,N, m\in{\mathbb N}}$, a ${N}-$dimensional  ${m}-$\textit{variable word} on ${k}$ letters is an element of ${\left([k]\cup\{x_1, \cdots,x_m\}\right)^N\setminus [k]^N}$ that uses all the symbols ${x_i}$ at least once.

Given ${k,n\in{\mathbb N}}$, we call $x-string$ (or ${n}-$dimensional \textit{variable word} on ${k}$ letters),  a finite word $a_1a_2\cdots a_n$ of the symboles $a_i \in [k] \cup \{x\}$, where at least one symbol $a_i$ is $x.$ $w(x)$ denotes an $x-string.$ Let $V$ denote  the set of all string: $V=\{w(x)\}$. The cardinal of $V$ is: $V=(k+1)^n-k^n.$ 

For any integer $i \in [k]$ and $x-string$ $w(x)$, the string obtained from $w(x)$ by replacing each $x$ by $i$ is denoted by $w(x;i).$ 
A \textit{combinatorial line} is a set of $k$ strings $\{w(x;i): \  i\in [k] \}$ where $w(x)$ is an $x-string$ \citep{beck2008combinatorial}. That is a combinatorial line is a set of $k$ finite words obtained by replacing $x$ in the word $w(x;i)$ by $i \in \{1,2, \cdots k\}.$

For instance, for $k=3$ and $n=8$, a combinatorial line is :

 $\{w(x)=1xx2x23x: \ x\in [3]\}=  \{11121231, 12222232, 13323233 \}.$

Sets which do not contain any combinatorial lines is called a \textit{line-free}. 

\begin{thm}[Hales-Jewett theorem]   For every pair of positive integers $k$ and $r$ there exists a positive number $HJ(k, r)$ such that for every $n \geq HJ(k, r)$ and every $r-$colouring of the set $[k]^n$ there is a
monochromatic combinatorial line.   \label{hj1} 	\end{thm}

There are several proofs of Hales-Jewett theorem. The original proof has been given by \cite{hales1987regularity}. \cite{shelah1988primitive} proved through primitive recursive bound using simple induction. \cite{nilli1990shelah} presented a compact form of Shelah’s Proof of the Hales-Jewett Theorem. \cite{matet2007shelah} presented a variant of Shelah’s proof of the Hales–Jewett theorem by replacing Shelah’s pigeonhole lemma by an appeal to Ramsey’s theorem.

The Hales-Jewett's theorem has also a density version. By considering a nonempty subset  $A$ of the set $[k]^n$, the density of $A$ inside $[k]^n$ is a positive real number $\delta=\frac{|A|}{k^n}$. Values of $\delta$ are bounded by $0$ and $1$, that is $0< \delta \leq 1.$ 

Let denote by $DHJ(k, \delta)$ the density Hales-Jewett number. The density version of Hales-Jewett theorem is announced as follows:

\begin{thm}[Density version of Hales-Jewett theorem]   For any $k \in \mathbb{Z}^+$ and any real number $0< \delta \leq 1$,  there exists a positive integer $DHJ(k, \delta)$ such that if $n \geq DHJ(k,\delta)$ and $A$ is any subset of $[k] ^n$ with $|A| \geq \delta k^n$, then $A$ contains a combinatorial line.  \label{hj2}	\end{thm}

The proof of the density version of Hales-Jewett theorem has been demonstrated by \cite{furstenberg1991density} using ergodic methods. \cite{Polymath2009new} gave an elementary non-ergodic proof of the density version of Hales-Jewett theorem by using the equal-slices measure. A simplified version of \cite{Polymath2009new} has been given by \cite{dodos2013simple} using  a purely combinatorial proof of the density Hales–Jewett Theorem.

To show that this density version of Hales-Jewett implies the Hales-Jewett, we need only to set as in  \eqref{vsz}, $\delta=\frac{1}{r}.$ By $r-$colouring the set $[k]^n$, that is by partitioning to $r$ classes,  if $a_{max}$ is the set containing the maximum number then $|A_{max} \geq \frac{k^n}{r}=\delta k^n.$ Hence, according to \eqref{hj2}, $a_{max}$ contains a combinatorial line.

\subsection{Hales-Jewett theorem implies Van der Waerden's theorem.}

To show that the Hales-Jewett theorem implies Van der Waerden's theorem, we need only to show that combinatorial lines corresponds to the arithmetic progression.

In \eqref{vsz} we have shown that Szemeredi's theorem implies Van der Waerden's theorem. To show that the density version of Hales-Jewett theorem implies Van der Waerden's theorem, we need to show that the density version of Hales-Jewett theorem implies Szemerédi's theorem. Hence, by transitivity, the density version of Hales-Jewett theorem implies Van der Waerden's theorem. 

Thereupon, whatever the kind of Hales-Jewett theorem used to establish the implication, we need  only to show that the combinatorial line involves the arithmetic progression.

Let us  assume that the Hales-Jewett theorem is verified and show that the combinatorial line of $k$ elements contained to the subset $A$ corresponds to the arithmetic progression of length $k$.

We have defined $[k]$ as the set $\{1,2,\cdots, k\}.$ Instead to start by $1$, let us start by $0.$ In this part, $[k]$ expresses the set $\{0,1,\cdots, k-1\}.$ It is obvious that $[k]=\mathbb{Z}/k\mathbb{Z}.$

Let $n$ be the positive number of the Hales-Jewett theorem, then the set $[k]^n=(\mathbb{Z}/k\mathbb{Z})^n=\{(x_0,x_1, \cdots, x_{n-1}): \ x_i \in [k] \}$ has $k^n$ elements. Similarly, $[k^n]=\{0,1,\cdots, k^n-1\}$ has also $k^n$ elements. The set $[k^n]$ contains natural number (in base 10). While, elements of the set $[k]^n$ are the digits  in base$-k$ number system of the numbers $\{0,1,\cdots,k^n-1\}.$

Let us consider the bijection $f:[k]^n \longrightarrow [k^n]$ defines as follows:

$$f(y_0,y_1,\cdots, y_{n-1})=y_0+y_1k+y_2+k^2+\cdots+y_{n-1}k^{n-1}.$$

Let $w(x) \in ([k] \cup \{x\})^n\setminus [k]^n$ be an $x-tring.$ The combinatorial line generates by $w(x)$ is a set of $k$ elements.

The difference between two consecutive elements $w(x;i_1)$ and $w(x;i_2)$ of this combinatorial line is a constant. Let us call this constant $l=(l_0, l_1, \cdots, l_{n-1})= w(x;i_1)-w(x;i_2)$ with $i_1 > i_2.$

For $j\in \{0,1,\cdots, n-1\}$, $l_j$ has two values:

$l_j= \left\lbrace \begin{array}{ll}1 & \text{ if } l_j=x \\ 0 & \text{ otherwise}    \end{array} \right.$

Let $w(x;0)=(y_{0,0},y_{1,0},\cdots, y_{n-1,0})$ be the first element of the combinatorial line generated by $w(x)$ . Then, for $0\leq i \leq n-1$ an element $w(x;i)$ of the combinatorial line can be expressed as: $$w(x;i)=w(x;o)+il.$$

Let call by $a$ the image of $w(x;0)$ by $f$, that is $a=f(w(x;0))$ and by $d$ the image of $l$ by $f$, that is $d=f(l).$ We denote by $D$ the set $\{j: \ l_j=x\}.$ $d$ can be expressed as:
$$d=f(l)=l_0+l_1k+\cdots+l_{n-1}k^{n-1}=\sum_{j=0}^{n-1}l_jk^j= \sum_{j\in D} k^j.$$
Thus, $f(w(x;i))=a+id$, $a$ and $d$ fixed, $0\leq i \leq k-1$, the set $\{a+id: \ i\in [k]\}$ forms an arithmetic progression of length $k.$ So, for any combinatorial line of $k$ elements corresponds an arithmetic progression of length $k.$

We just need to take $N(k,\delta)=k^n$ to establish that the Hales-Jewett theorem implies the Szemerédi's theorem. As this latter implies the Van der Waerden's theorem. Similarly, we need to take $N(k,r)=k^n$ to show that the Hales-Jewett theorem implies the Van der Waerden's theorem.






















%The idea is to let  $ N_0=\ell^{n_0}$, and to associate with every number in $ [N]$ the digits in its expansion in the base-$ \ell$ number system. Then any combinatorial line would correspond to an arithmetic progression. (But not conversely. Why?)




