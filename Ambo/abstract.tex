
% Abstracts are usually written in English, with a version in your
% mother tongue underneath
\chapter*{Abstract} 
\addcontentsline{toc}{chapter}{Abstract}
% Don't change anything above this.

Games are inherent to  human nature and are present in all cultures. This research studies the class of mathematical games  called  \say{\emph{prover games}}
%\Jnote{s/mathematical game called prover games/class of mathematical games  called prover games}
by establishing the connection between parallel repetition of multi-prover games and the Hales-Jewett theorem. The umbilical cord that connects parallel repetition and the Hales-Jewett theorem is the combinatorial line.

%\Jnote{Make a second paragraph here.}
In this essay, firstly from the literature review 
%\Jnote{s/paper/essay}
four implications are proved between these theorems: Van der Waerden's theorem, Szemerédi's theorem, Hales-Jewett theorem and the density version of Hales-Jewett theorem. Then,  some notions about  two-prover games
%\Jnote{s/a two-prover game/two-prover games}
are  generalized. Eventually, the double implication between parallel repetition and the Hales-Jewett theorem is 
%\Jnote{s/are/is}
demonstrated. The proofs have been adapted from \cite{verbitsky1996towards} and \cite*{hkazla2016forbidden}. Namely, we used the result and the proof of \cite{verbitsky1996towards} to establish that the parallel repetition implies the Hales-Jewett theorem, and to show that the Hales-Jewett theorem implies the parallel repetition we used the result and the proof from \cite*{hkazla2016forbidden}.

% At a unviersity like Stellenbosch you *must* produce an abstract in Afrikaans for your masters.
% At AIMS you are encouraged to repeat the abstract in your mother tongue
% French, Igbo, Mlagasy, etc. just write it using LaTeX's special
% characters.
% Arabic students see the arabic.tex file for an example
% Amharic use openoffice and export from there and import a figure here.
% Where the words do not exist put the English work in italics, or use mathematical symbols.




