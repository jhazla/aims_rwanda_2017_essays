\chapter{Conclusion}

In this study, the relationship between parallel repetition and the density version of the Hales-Jewett theorem was analysed. We have shown that this umbilical cord which connects them is the combinatorial line. This study consisted of three parts  following our thesis statements. 

In the first place, we have started by investigating on
\Jnote{Delete ``on''}
the Van der Waerden's theorem,  the Szemerédi's theorem, the Hales-  Jewett theorem
\Jnote{No space after hyphen.}
and the density version of the Hales-Jewett theorem. We have proved 4 implications: the density version of the Hales-Jewett theorem implies the Hales-Jewett theorem and the Szemerédi's theorem, the Hales-Jewett is a generalisation of the Van der Waerden's theorem and the Szemerédi's theorem is only the density version of the Van der Waerden's theorem.

Also, we have generalized
\Jnote{s/generalized/presented}
some notions on prover games and on its parallel repetition. A summary of some known boundary of the value of a parallel repetition of two-prover games was given. We have given a proof which shows that the $n$-th power of the value of a game $G$  is less or equal to the value of the $n$-fold parallel repetition $G^n.$
\Jnote{Add: ``Furthermore, we showed that those two values do not have to be equal''.}
To support it we have constructed  an example.

Lastly, we have extended the proof of Oleg Verbitsky from two provers to multiple provers by showing that the value of the parallel repetition of multi-prover games is bounded above by the density Hales-Jewett number. Specifically, we have shown that the density version of the Hales-Jewett theorem implies the parallel repetition. Inversely, we have constructed a game which shows that the value of the parallel repetition of multi-prover games is bounded below by the density Hales-Jewett number. In other words, we have established that the parallel repetition implies the density version of the Hales-Jewett theorem for a multi-prover game that we have constructed.
\Jnote{Please cite appropriate papers and make clear that this is not your
  original work.}

A general exponential decay bound like the Raz theorem for parallel repetition of  multi-prover games is still not known. Future research may focus on: generalizing this bound for multi-prover games or simplifying the Raz proof, and showing or disproving the existence of the general exponentially decay
\Jnote{s/decay/decaying}
bound. Equally,  exact general expressions of the Van der Waerden number, the density Szemerédi's number and the density Hales-Jewett number are not known and form open problems for research. Even showing the existence of these numbers is still a problem.
\Jnote{It is not an open problem, but the proofs are difficult.}
Further work should also focus on understanding and summarizing papers that apply parallel repetition to hardness of approximation and exploring in depth the parallel repetition of quantum games.