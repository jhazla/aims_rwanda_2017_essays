

\chapter{Connection between parallel repetition of multi-prover games and  Hales-Jewett theorem.}

This chapter presents the relationship between parallel repetition of multiple provers with the density Hales-Jewett theorem. We give a parallel repetition bound using the density Hales-Jewett. Firstly, we show that the density Hales-Jewett theorem implies parallel repetition. Secondly, we show that the parallel repetition implies the density Hales-Jewett theorem.

\section{Hales-Jewett theorem implies parallel repetition.}

In both versions of Hales-Jewett theorem (see \eqref{hj1} and \eqref{hj2}), the concept which emphasizes this theorem is the \textit{combinatorial line.} The combinatorial line is the umbilical cord between the Hales-Jewett theorem and the parallel repetition. In section \eqref{hjt}, we have already explain in a detailed way
%\Jnote{s/deeply/in a detailed way}
and define what the combinatorial is. Let us recall some notions about a combinatorial line and the formulation of the Hales-Jewett theorem.

Let $k, n\in \mathbb{Z}^+$, $[k]=\{1,2, \ldots,k\}$ and an $x-$string $w(x)=a_1a_2\ldots a_n \in ([k]\times\{x\})^n\setminus [k]^n.$ That is, in $w(x)=a_1a_2\ldots a_n$, at least one of the symbol   $a_i$ contains the symbol  $x$ called wildcard. Let $w(x;i)$ be the string obtained by replacing $x$ by $i$.

The \textit{combinatorial line} is the set of $k$ strings $\{w(x;i): \ i\in \{1,2,\ldots,k\} \}$, that is the set $\{w(x;1), w(x;2), \ldots, w(x;k)\}.$ 
 
The Hales-Jewett theorem is given in \eqref{hj1}.
%\Jnote{Don't start sentence with ``So''. You can say ``HJ theorem is given in ()''.}
As the name stipulates, the Hales-Jewett was proved by Hales and Jewett in 1963. The formulation is based on colouring of a set and on the existence of  a monochromatic combinatorial line.

Furthermore, there is a density formulation of Hales-Jewett theorem on which this section is mainly constructed. Given a subset $A$ of $[k]^n$, the density of $A$ is defined and denoted as $\delta(A)=\frac{|A|}{k^n}.$ By simplicity, $\delta$ denotes the density of $A$, that is $\delta=\delta(A).$

Thereby, the density version of Hales-Jewett theorem states that for any positive number $k$ and real number $\delta$,  there exists a large enough number $n$ (depending on $k$ and $\delta$) such that  any subset of  $[k]^n $ with density $\delta$ contains a combinatorial line.
%\Jnote{It is not necessary to repeat concepts from the previous chapter in   such detailed way.}
In the following, essentially  we use  the density version of  Hales-Jewett theorem. Whenever we say the Hales-Jewett theorem, we mean the density version of Hales-Jewett theorem.
%\Jnote{s/Whenever there is/Whenever we say} 

We denote by $\Delta_{k,n}$ the maximum density of a subset $W$ of $[k]^n$ without a combinatorial line. $\Delta\_{k,n}$ was discussed in  \eqref{dhjn}.
%\Jnote{Ref is broken. s/We have discussed a lot/Delta\_kn was discussed in}
The number $\Delta_{k,n}$ is called density Hales-Jewett number.

The density version of the Hales-Jewett theorem is equivalent to  $\lim\limits_{n\longrightarrow \infty} \Delta_{k,n}=0$ for $k\geq 2$ \citep{furstenberg1991density}. Let us show it. \begin{itemize}
\item The density version of the Hales-Jewett theorem implies   $\lim\limits_{n\longrightarrow \infty} \Delta_{k,n}=0$. We assume that the density version of the Hales-Jewett theorem is true, that is $\forall k, \delta, \exists DHJ(k,\delta)  \in \mathbb{N} / \forall n\geq DHJ(k,\delta)$ and $\forall S \subseteq [k]^n, |S| \geq \delta k^n$, $S$ contains a combinatorial line. This means that  there is a subset $W \subseteq [k]^n$ which does not contain a combinatorial line with density $|W| < \delta k^n \Longleftrightarrow |\frac{|W|}{k^n}-0| <\delta \Longleftrightarrow |\Delta_{k,n}-0|<\delta $ . So, $\lim\limits_{n\longrightarrow \infty} \Delta_{k,n}=0$.
\item $\lim\limits_{n\longrightarrow \infty} \Delta_{k,n}=0$ implies the density version of the Hales-Jewett theorem. $\lim\limits_{n\longrightarrow \infty} \Delta_{k,n}=0$ is equivalent to $\forall , \delta >0, \exists N_0 \in \mathbb{N} /\forall n \geq N_0, \Delta_{k,n} < \delta$  fro a fixed $k.$ $ \Delta_{k,n}$ is the density of the largest subset of $[k]^n$ without a combinatorial line. Hence, $\forall n \geq N_0, \forall S \subseteq [k]^n$ with $|S| \geq \delta k^n$ contains a combinatorial line.
\end{itemize} 

%\Jnote{It is still misleading. It was not demonstrated ``during their work on DHJ''.  It \emph{is} the DHJ! I mean it is easy to see that it is equivalent to Theorem 2.4.2.Please say this.}

%\begin{thm}[\cite{furstenberg1991density}]	For $k\geq 2$, $\lim\limits_{n\longrightarrow \infty} \Delta_{k,n}=0.$ \label{fk}	\end{thm}
%\Jnote{Be clear that this \emph{is} the density Hales-Jewett theorem!   Hales-Jewett was proved by Hales and Jewett and the density version   (density Hales-Jewett) was proved by Furstenberg and Katznelson.}
 
This result shows that  a subset of $[k]^n$ with constant density 
%\Jnote{s/the set/a subset of $[k]^n$ with constant measure}
will  necessarily contain  a combinatorial line when $n$ increases.
%\Jnote{almost necessary/necessarily}
%\Jnote{This paragraph belongs to previous chapter.}
`

The density Hales-Jewett number converges to $0$ when $n$ converges to infinity. Equally, the Raz theorem \eqref{prt} shows that the value of a parallel repetition of a game (non-trivial) decreases exponentially  fast to $0$ when $n$ converges to infinity.  Note that the convergence of the Raz theorem is fast than the convergence of the  density Hales-Jewett number.
%\Jnote{No, it is not. The theorems are uncomparable. If you take very large
 %nswer alphabet size Raz becomes useless while Verbitsky still works.
  %If you take constant alphabet size and epsilon Raz theorem is much faster.}
  
The following Oleg Verbitsky theorem  shows that the density Hales-Jewett theorem implies the parallel repetition of multi-prover games.

\begin{thm}[\cite{verbitsky1996towards}]	 Let $G$ be a non-trivial multi-prover game with $|Q|=r$ the size of question set. Then, 
  $\val (G^n) \leq \Delta_{r,n}.$	\label{ver96} \end{thm}

Knowing that the density Hales-Jewett number converges to $0$ when $n$ converges to infinity, then we obtain the following consequence.
\begin{cor}	Let $G$ be a non-trivial multi-prover game. Then, $\lim\limits_{n\longrightarrow \infty} \val (G^n)=0.$ 	\end{cor}

The theorem \eqref{ver96}  has been proved by \cite{verbitsky1996towards} for two-prover games.  His proof can be extended for  multi-prover games in our case,  that is, for $k$ players with $k\geq 2.$ To establish the truth of  this theorem, Oleg Verbitsky used the proof by contradiction. The general idea is: given a subset $K$ of $Q^n$ for which the provers win   for a given strategy, we must show that $K$ is  the subset  of $Q^n$ without a combinatorial line.
%\Jnote{Say that $W$ is the subset of $Q^n$ for which the provers win   for a given strategy.} 
So, we assume that there is a combinatorial line and then we show that there is  contradiction.

Let us adapt our proof from the proof of  \cite{verbitsky1996towards} to show the theorem \eqref{ver96} for multi-prover games, that is, we extend the proof of Oleg Verbitsky from two-prover games to multi-prover games.

\begin{proof}
  Let $G$ be a $k-$prover game (non-trivial), that is $G(\phi, Q\subseteq X^1 \times \ldots \times X^k, A^1 \times \ldots \times A^k, \mu)$ where $X^t$ and $A^t$ represent respectively the set of questions and the set of answers of the player $t$, for $1\leq t \leq k.$ The set $Q$ is a subset of the set $X^1 \times \ldots \times X^k$ where elements are chosen uniformly  according to the probability distribution $\mu$.
 % \Jnote{For this proof you have to assume that $\mu$ is uniform.}

 Let $|Q|=r$, with $Q=\{q_1, \ldots, q_r\}$ where $q_j=(q_j^1,\ldots, q_j^k)$, $q_j^t \in X^t$ for $j\leq r.$ The superscript  $t$ highlights the component (player), while the subscript $j$ denotes the number (order) of questions. For instance the question $q_j^t$ is the $j-$th question addressed to the player number $t.$  For the parallel repetition $G^n$, let us consider $F^1, \ldots, F^k$ like the $k$ optimal strategies of the game where each strategy is an $n-$tuple function of strategies, that is $F^t=(f_1^t,\ldots, f_n^t)$. We denote by $K$ the set of success questions using these strategies in $G^n.$ The set $K$ can be expressed as: 
 
 $K=\{(s_1, \ldots, s_n) \in Q^n: \bigwedge\limits_{i=1}^n \phi \left[ s_i^1, \ldots, s_i^k, f_i^1(s_1^1, \ldots, s_n^1), \ldots, f_i^k(s_1^k, \ldots, s_n^k) \right]=1 \}.$
 %\Jnote{Before the proof you used $W$, now you use $K$. Be consistent.}

Note that for $1\leq i \leq n$,  $s_i \in Q=\{q_1, \ldots, q_r\}.$ $s_i^t$ denotes an $i-$th question in parallel repetition addressed to the player $t$. This question can be any of the $t-$th component of the set  $q_j.$
 
As $K$ is the set of success questions, then the value of the game $G^n$ is: $\val (G^n) = \frac{|K|}{r^n}.$
%\Jnote{Note that this is the place where you use uniform distribution assumption}.
 
In this stage, we can not say that $\Delta_{r,n} \geq \frac{|K|}{r^n}$
%\Jnote{Not equal, it should be $\ge$.}
because we do not know if the set  $K$ does not contain any combinatorial lines. Let us show that $K$ is a set without a combinatorial line.

Let us suppose by contradiction that there is a combinatorial line $L=\{\bar{b}_1, \ldots, \bar{b}_r \} \subseteq K.$ In this case,  the game  $G$ should be trivial.
%\Jnote{Game $G$ has nothing to do with the proposition.}
 
Let $C=C_1\ldots C_n$ be an $r \times n$ matrix whose $r$ rows are $\bar{b}_1, \ldots, \bar{b}_r$ and $n$ columns $C_1\ldots C_n$ each are either $(q_j,q_j,\ldots,q_j)^T$ for some $j\leq r$ or $(q_1,q_2,\ldots,q_r)^T.$ By definition of a combinatorial line, there exists at least one column $C_l=(q_1,q_2,\ldots,q_r)^T.$ We assume that $L$ is ordered so that the intersection of the row $\bar{b}_j$ and the column $C_l$ of the matrix is the element $q_j.$ The element $q_j=(q_j^1,\ldots, q_j^k)$ has $k$ components. So, the  matrix $C$ can be expanded to the $kr \times n$ matrix $D$ by replacing each matrix element $q_j$ with the column $(q_j^1,\ldots, q_j^k)^T.$ There are $kr$ rows of the matrix D and $n$ columns. Thus, let us denote by $\bar{x}_1^1, \ldots, \bar{x}_1^k, \ldots, \bar{x}_r^1, \ldots, \bar{x}_r^k$ the rows of the matrix $D$ where $\bar{x}_j^t \in  (X^t)^n.$

Since $L$ is a combinatorial line, let us use one of the strategy of the matrix element in the column $C_l$ which is in the form $(q_1,q_2,\ldots,q_r)^T.$  Note that $q_j$ is a $k-$tuple. 
Let us define strategies $f^1,f^2, \ldots, f^k$ in the game $G$ by $f^t(q^t)=f_l^t(\bar{x}_{n_t}^t)$ where $x_{n_t}^t=q^t$ for $1\leq t \leq k.$ 
%These strategies $f^t$ are well defined, since for distinct such $n_t$ and $n_t'$ it holds $\bar{x}_{n_t}^t= \bar{x}_{n_t'}^t.$
%\Jnote{I don't understand last sentence.}

For  arbitrary $q_j= (q_j^1,\ldots, q_j^k) \in Q$, we have:
$$\phi (q^1,\ldots, q^k, f^1(q^1), \ldots, f^k(q^k))= \phi (q_j^1,\ldots, q_j^k, f_l^1(\bar{x}_j^1), \ldots, f_l^k(\bar{x}_j^k))=1$$

As $b_j \in K$, strategies $F^1, \ldots, F^k$ win in  the $l-$th copy of $G$. That is the game $G$ is  trivial. 
%\Jnote{It \emph{is} trivial.}

Hence, there is a contradiction with our assumption that $K$ contains a combinatorial line.

Therefore, $K$ does not contain a combinatorial line and $\Delta_{r,n} \geq  \frac{|K|}{r^n}$.
%\Jnote{Change $=$ to $\le$.}

It results that $\val (G^n) \leq \Delta_{r,n}.$
\end{proof}

Let $\nu_{Q,n}=\max_G \val (G^n)$ where the maximum is over all non-trivial games $G$ with $|Q|=r$ the size of the set of questions $Q.$ 
The Oleg Verbitsky's theorem \eqref{ver96} is applicable to $\nu_{Q,n}$, that is $\nu_{Q,n} \leq \Delta_{r,n}.$
%\Jnote{What is $r$?}
Then, $\lim\limits_{n\longrightarrow \infty} \nu_{Q,n}=0.$

\section{Parallel repetition implies Hales-Jewett theorem.}
 
To show that the parallel repetition implies the Hales-Jewett theorem, let us firstly define a set of questions on which will be constructed some multi-prover games.

\begin{defn}Let $k\geq 2$ and $Q_k \subseteq \{0,1\}^k$ a question set of size $k.$ An \textit{k-prover question set} is a question set $Q_k$ where the $t-$th question contains $1$ in the $t-$th position and $0$ in the remaining positions. This question set can be expressed as:
$$Q_k=\left\lbrace(q^1, \ldots, q^k): |\{t:q^t=1\}|=1\right\rbrace.$$		\end{defn}
 
 An extensional definition of the question set $Q_k$ is: $Q_k=\left\lbrace (1,\ldots,0), (0,1,\ldots,0), \ldots, (0,\ldots,1) \right\rbrace.$ $|Q_k|=k$ and the elements of the question set $Q_k$ are equivalent to the elements of the canonical basis, that is $Q_k= \left\lbrace e_1, e_2, \ldots, e_k\right\rbrace$ where $e_l=(\delta_{1l}, \delta_{2l}, \ldots, \delta_{kl} )$, $\delta_{ml}$ is the Kronecker delta which equals to $1$ if $l=m$ and $0$ whenever $l \neq m$ for $1 \leq l, m \leq  k.$ 

The following theorem highlights that there exists a game such that the parallel repetition of this game  implies the density Hales-Jewett theorem. This result announced as theorem \eqref{hka} links the existence of a combinatorial line in a set with the parallel repetition value of a certain game.

%\Jnote{Use \textbackslash cite* to cite.}
 \begin{thm}[\cite*{hkazla2016forbidden}] Let $k\geq 3$, $n\geq 1$ and $S\subseteq [k]^n$ with density $\delta=|S|/k^n$ such that $S$ does not contain a combinatorial line.	
 
There exists a $k-$prover game $G_S$ with question set $Q_k$ and with answer alphabets,
$A^t = 2^{[n]} \times [n]$ such that:
\begin{itemize}
\item $\val (G_S) \leq 1-1/k.$ 	\item $\val (G_S^n) \geq \delta(S).$
\end{itemize} \label{hka}
 	\end{thm}

Thus, from the theorem \eqref{hka} we can deduce the value of the $n-$fold parallel repetition $G_S^n$ when $S$ is the maximum subset of  $S\subseteq [k]^n$ without a combinatorial line, that is when the density of $S$ is $\Delta_{k,n}= |S|/k^n$ where $k\geq 3$, $n\geq 1$. This result given as theorem \eqref{hka1} is complementary to  Oleg Verbitsky theorem \eqref{ver96}.

 \begin{thm} Let $k\geq 3$, $n\geq 1$ and $S\subseteq [k]^n$ with density $\Delta_{k,n}$. We have: 	
 $\val (G_S^n) \geq \Delta_{k,n}.$  \label{hka1}
 	\end{thm} 
For this  game $G_S$, according to the theorems \eqref{ver96} and \eqref{hka1}, we conclude that $\val (G_S^n)=\Delta_{k,n}.$

To prove  the theorem \eqref{hka}, we need to construct a game which  satisfies the conditions on theorem \eqref{hka}. So, let us construct a game $G_S$   as  defined by \cite{hkazla2016forbidden} based to the subset $S $ of the set $[k]^n$. 

Let $k \geq 3$, $n \geq 1$ and $S \subseteq [k]^n$ with $\delta (S)= \frac{|S|}{k^n}$. The game $G_S$ with question set $Q_k$ which we will define must satisfy the  following requirements: 
\begin{itemize}
\item If $S$ does not contain a combinatorial line, then $G_S$ is non-trivial. \item $\val (G_S^n) \geq \delta (S).$
\end{itemize}

As $|Q_k|=k$ and $|[k]|=k$, there is a natural bijection between the question tuples in $Q_k$ and $[k]$. So, the game $G_S$ is played as this. The verifier chooses the number of a special prover $t \in [k]$   and sends $1$ to the special prover and $0$ to all other provers. 
The answer set of the game $G_S$ is the same for all provers: $A^t=2^{[n]} \times [n]$ where the power set  $2^{[n]}$ denotes the set of all subsets of $[n]$. Note that the set $2^{[n]}$ is equivalent to the set $\{1,2,\ldots, 2^n\}.$ Thus, answers from provers are in the form $(T^1, z^1), \ldots, (T^k, z^k)$. The verifier checks the following conditions and accepts if all of them are met: 
\begin{itemize}
\item The sets $T^1, T^2,\ldots, T^k$ form a partition of $[n].$
\item $z^1=z^2=\ldots=z^k=z.$
\item $z \in T^t$
\item Let $\bar{s} = (s_1, s_2, \ldots, s_n)$ be the string over $[k]^n$ such that  $s_i = e$ if and only if $i \in  T ^e$ for $1\leq i \leq n.$ Then, $\bar{s} \in S.$
\end{itemize}

From the definition of the game $G_S$ we can deduce the following propositions given and proved by \cite{hkazla2016forbidden}. So, the proofs of these propositions are adapted from this latter paper.  
\begin{pro} If $S$ has a combinatorial line, then the game $G_S$ is trivial	. \label{pr1}	\end{pro}
%\Jnote{You have to define $G_S$ before stating the proposition.}
\begin{proof}
We assume that $S \subseteq [k]^n$ has a combinatorial line. 
Let $\bar{b}=w(x)=(b_1,\ldots, b_n)$ an $x-$string for which the  combinatorial line is $L(\bar{b})=\{w(x;i): i \in [k]\} \subseteq S$ and let fix a position $z \in [n]$ with $b_z=x$. Note that $b_1,\ldots, b_n \in [k] \cup \{x\}$. For $p \in [k] \cup \{x\}$, let us define a set $B(p)$ as: $B(p)=\{j: b_j=p\}$. The set $B(p)$ is the set of coordinates $j$ in which $b_j$ equals  to $p.$ Now, let us define the  strategy for which prover $e$ will use to answer questions:
$$ f^e(q^e)=\left\lbrace \begin{array}{ll} (B(e),z) & \text{if } q^e=0, \\ (B(e) \cup B(x), z) & \text{if } q^e=1.\end{array} \right. $$

Thus, the verifier checks the four conditions. The four conditions are all satisfied. In effect, the  first condition will be  always accepted by the verifier   because the sets $B(1), \ldots, B(k), B(x)$ from a partition. All $z^e$ are equal, that is $z^1=\ldots=z^k$, then the second condition is satisfied. Because the prover $t$ responds with  $(B(t) \cup B(x), z)$ and $z \in B(x)$, then $z \in T^t$: the third condition is satisfied. The fourth condition is also satisfied because $\bar{s}=\bar{b}$ and $\bar{s}=\bar{b} =w(t) \in L(\bar{b}) \subseteq S.$
\end{proof}

\begin{pro} If the game $G_S$ is trivial, then $S$ has a combinatorial line.	 \label{pr2}	\end{pro}

\begin{proof}
We assume that the game $G_S$ is trivial. As the game $G$ is trivial, there is a $k$-tuple of strategies for which the provers  always win.  Let $f^1, \ldots, f^k$ be this $k$-tuple of  strategies. The form of the answer of the prover $e$ to the question $q \in \{0,1\}$ is similar as in the definition of the game $G_S$ and is defined as: $(T_q^e, z_q^e)=f^e(q)$ where $e \in [k]$.  As the game is trivial we have $z_0^1=z_0^2= \ldots=z_0^k=z_1^1=z_1^2= \ldots=z_1^k=z$. 
For any two $e\neq e'$, $T_0^e \cap T_0^{e'}= \emptyset$. If $t \neq e$ and $t\neq e'$, the verifier will reject.
$z \notin T_0^1 \cup \ldots \cup T_0^k$, because if $z \in T_0^e$, the verifier rejects if $t \neq e$. Therefore, the word $\bar{b}=w(x)$ (combinatorial line)  is defined as: $$b_i=\left\lbrace \begin{array}{ll} e & \text{if } i \in T_0^e, \ \text{for }e \in [k], \\ x & \text{ otherwise}.\end{array} \right.$$
For a fix $t \in [k]$, let us show that $w(t) \in S.$ By picking the special prover $t$, the verifier checks that the sets $T^e$ form a partition. In this case the answer of the prover $t$ is $T_1^t=[n] \setminus \left(T_0^1 \cup \ldots \cup T_0^{t-1} \cup T_0^{t+1} \cup \ldots T_0^k	 \right) .$ For every $t$, $w(t) \in S.$ Hence, $L(\bar{b}) \subseteq S.$
\end{proof}

\begin{pro}	 The value of $G_S^n$ is at least $\delta(S)$	  \label{pr3} \end{pro}

\begin{proof}
For $1\leq e \leq k$, let $q^e$ be the question. Let $n \geq 1$ and $G^n$ be the $n$-fold parallel repetition. Note that question $q^e$ in the game $G^n_S$ is an $n$-tuple defined as: $q^e=(q_1^e, q_2^e, \ldots, q_n^e)$ where for a fixed $i \in \mathbb{N}$ there is necessary  one special $t$ for which $q_i^t=1$ and other $q_i^e=0$.  In other words, $(q_i^t)$  forms a  $k \times n$ matrix where in each column there is at most one element equals to $1.$ But, for a fixed $t$ there is at least one special $i$ for which $q_i^t=1$. In other words, in a line of the $k \times n$ matrix there is at least one one, that is the cardinality of the set $\{i \in [n]: q_i^e=1\}$ is at least one.

Let  $T^e=\{i \in [n]: q_i^e=1\}$. in coordinate (column) $i$ the prover $e$ responds with $(T^e,i).$ For $1\leq e \leq k$ and $1\leq i  \leq k$, let  $(a_i=t)$ where $q_i^t=1$, that is $a_1, \ldots, a_n$ form a sequence of special provers.  Then for a fixed $i$, sets $T^1, \ldots, T^k$ form a partition of $[n].$ Equally, $z_i^1=\ldots=z_i^k.$ Also, $i \in T^{a_i}$. Finally, $\bar{s}=(a_1, \ldots, a_n) \in S.$ Let us compute the value of $G_S^n$. The string $a_i$ can take $k$ different values. So the probability for $a_i$ to be a string of $\bar{s}$ is $\frac{1}{k^n}$. Now, the probability of $\bar{s}$ to be in $S$ is:  $\Pr (\bar{s} \in S)=  \frac{|S|}{k^n}=\delta(S).$ Hence, $\val (G_S^n) \geq \delta (S).$

\end{proof}

%\Jnote{The language you use in these three propositions cannot be so close to
%  what we have in our paper. You should write these proofs
%  \emph{in your own words}. This means while writing you 
%  \emph{should not be looking at my paper}.}



%A short proof of the theorem \eqref{hka} has been given by \cite{hkazla2016forbidden} by using the proposition \eqref{prop} and theorem \eqref{hhm} which contain  the notion of \textit{homomorphism} of question sets. Let us introduce notions of homomorphism.
%
%Let $k\geq 2$ and $Q \subseteq X^1 \times \ldots \times X^k$ be a $k-$prover question set. Consider the $r-$regular, $r-$partite hypergraph\footnote{A hypergraph is pair $(X,E)$ where $X$ is a set of elements called \textit{nodes} or \textit{vertices}, and $E$ is a set of non-empty subsets of $X$ called hyperedges (set of nodes) or edges. For further reading, see https://en.wikipedia.org/wiki/Hypergraph} $G=(X^1 \times \ldots \times X^k, Q).$
%
% Given two hypergraphs $(X^1 \times \ldots \times X^k, Q)$ and $(Y^1 \times \ldots \times Y^k, P)$. The function $f=(f^1, \ldots, f^k)$ where $f^t: X^t \longrightarrow Y^t$ is a homomorphism from $Q$ to $P$ if $\bar{q}=(q^1, \ldots,q^k) \in Q$ implies $f(\bar{q})=(f^1(q^1), \ldots,f^k(q^k)) \in P.$
%
%Let $S\subseteq Q^n$ with $\delta(S)= |S|/|Q^n|$ the density of $S$, and $f=(f_1, \ldots, f_n)$ be a vector of $n$ homomorphisms of $Q$ (from $Q$ to $Q$). $f$ is \textit{good} for $S$ if:
%\begin{itemize}
%\item For every $\bar{q}=(q^1, \ldots,q^k) \in Q$, we have $f(\bar{q})=(f_1(\bar{q}), \ldots, f_n(\bar{q})) \in S.$
%\item There exists $i \in  [n]$ such that $f_i$ is identity.
%\end{itemize}
%
%\Jnote{Please summarize the direct proof, not the one with homomorphisms.}
%
%\begin{thm}[\cite{feige1996error}]	Let $Q$ be a connected, $k-$prover question set and $S \subseteq Q^n$. There exists an $k-$prover game $G_S$ with question set $Q$ such that:
%\begin{itemize}
%\item If $G^S$ is trivial, then there exists a homomorphism vector $f$ that is good for $S$.
%\item $\val (G_S^n) \geq \delta.$
%\end{itemize} \label{hhm}	\end{thm} 
%
%\begin{pro}	Let $r \geq 3$ and $S \subseteq Q_k^n \cong [r]^n$ such that there exists a homomorphism vector $f$ that is good for $S$. Then, $S$ contains a combinatorial line.	\label{prop}\end{pro}
% 
%Moreover, let us consider that $S$ is a subset of $[k]^n$ without a combinatorial line. Assume that $S$ is the maximum subset of $[k]^n$ without a combinatorial line, then from theorem \eqref{hka} we obtain the theorem \eqref{hkb} which is a  complementary inequality to theorem \eqref{ver96}.
% 
% \begin{thm}[\cite{hkazla2016forbidden}] For $k \geq 3$, $\Delta_{k,n}	\leq \val (G^n).$ \label{hkb}	\end{thm}
% 
% Considering that $\nu_{Q,n}=\max_G \val (G^n)$ where the maximum is over all non-trivial games $G$ with question set $Q$, we have $\Delta_{k,n}	\leq \nu_{Q,n}$ which remains true.
% 
% By combining the two theorems \eqref{ver96} and \eqref{hkb}, we have $\val (G^n)=\Delta_{k,n}.$. Likewise, the maximum  value of  all non-trivial games equals to the density of the maximum subset of $[k]^n$ without a combinatorial line, that is $\nu_{Q,n}=\Delta_{k,n}.$
% 
%From the best known lower bound of $\Delta_{k,n}$ established by \cite{polymath2010density}, we can apply it to bound $\val (G^n)$. This lower bound adapted by \cite{hkazla2016forbidden}, thereafter for this kind of multi-prover is formulated in \eqref{hkc}
% 
% \begin{thm}Let $\ell \geq 1$ and $k=2^{\ell-1}+1.$ There exists $C_{\ell} >0$ such that for every $n\geq 2$ there exists a set $S \subseteq [k]^n$ with $$ \delta(S) \geq \exp \left( -C_{\ell} (\log n)^{1/\ell} \right)$$	 such that $S$ does not contain a combinatorial line  \label{hkc}	\end{thm}
%
% As $\val (G_S^n) \geq  \delta(S)$, we deduce from \eqref{hkc} that  $\val (G_S^n) \geq  \exp \left( -C_{\ell} (\log n)^{1/\ell} \right)$ for $k=2^{\ell-1}+1.$
%
% \Jnote{The section with proofs seems disorganized. Please divide it clearly
%   into two parts (DHJ => PR) and (PR => DHJ) without mixing them up.}
% 