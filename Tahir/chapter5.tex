\chapter{Conclusion}
In this essay, we used different approaches to explore quantum mechanics non-locality.
First, we showed that quantum advantage can be vary for binary games, for some games quantum strategy can be better that classical version. An example is the $CHSH$ game, but quantum advantages is also limited by upper bound.
We also demonstrate the  sensitivity of quantum advantage to winning conditions of the game, by slight modification of the game rule we lose  advantage of quantum mechanics. An example is what happen when we  replaced $XOR$ by $NAND$ in winning conditions the $CHSH$ game.
Secondly, we characterised quantum mechanics violation of local realism for maximally entangled state of two qubits, by steering a subsystem to two and three ensembles of orthogonal bases.

Further investigation could be held in direction of classifying the games, according to their ability of giving quantum advantages, this would be answer many questions about the computational benefit of quantum world. For the quantum mechanics violation of local realism via steering we could investigate what happen when the state has more dimensions or for different degrees of entanglement of initial state.
%\begin{itemize}
%%\item What should be the structure of game to gain quantum advantage?
%%\item is it possible to classified all the binary games according to quantum strategies?
%\item what happen if the state steered to more ensembles ?
%\item  what happen if use different state?
%\item  how about adding more qubits?
%\item what happen if  state is mixture, pure or non entangled?
%\end{itemize}
% 