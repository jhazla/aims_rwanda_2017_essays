\chapter{Quantum Non-Locality via Steering Theorem}
Our motivation is built on work of Jevtic and  Rudolph \citep*{Jevtic:2015:10.1364/JOSAB.32.000A50}, where they used a simple approach to show that  quantum mechanics violates local realism in the form of hidden variables theory. The impossibility of local hidden variables interpretation for quantum mechanics was illustrated in 1964 by Bell in his celebrated theory  \citep*{book:800289}. Jevtic and  Rudolph, make a different argument that can be seen as consistency simpler.

\section{Introduction}\hfill \break
Consider two players space-like separated, Alice and Bob, who share a bipartite quantum state  $\ket{\Psi}$ with reduced states $\ket{\Psi_A}$ and $\ket{\Psi_B}$ for the two parties, respectively. Alice can perform a collection of local measurements from the set of observables $\{M_A\}$ for her part. Each local measurement by Alice steers Bob part to new state \citep{book:474706}. To illustrate more, consider Alice has two sets of orthogonal bases , namely $\{\ket{u_n}\}$ and $\{\ket{v_n}\}$.  Then $\ket{\Psi}$ can be decomposed as,
\begin{equation}
\ket{\Psi}=\sum_n c_n  \ket{u_n}\ket{\phi_n}=\sum_n d_n\ket{v_n} \ket{\varphi_n}.
\end{equation}
The fact that $\ket{\phi_n} \text{ and} \ket{\varphi_n}$ are different state for Bob's part were used by EPR in their argument for hidden variables theory, as well as in their claim about incompleteness of quantum mechanics.

 In 2007, Wiseman, Jones and Doherty formalised steering in terms of the incompatibility of quantum mechanical predictions with a classical-quantum model where pre-determined states are sent to the parties \citep*{PhysRevA.76.052116}. Furthermore, the in of quantum steering can also be seen as an indication of entanglement as well as the impossibility of local hidden variables description \citep*{Jevtic:2015:10.1364/JOSAB.32.000A50}.


In general, in any hidden variable theory, each qubit has an assigned value for each observable, determined by a hidden variable (or a set of hidden variables) of real
states $\lambda$. A statistical ensembles of particles has a certain distribution $x(\lambda)$ of the hidden variable and thus the average value of an observable $A$ is given by
\begin{equation}
\langle A \rangle =\int d \lambda A(\lambda),
\end{equation}
where $A(\lambda)$ is expectation value of $A$ on the real state $\lambda$.


\section{Experiment}\hfill \break
In the following, we describe experiment used by Jevtic and Rudolph to demonstrate the violation of quantum mechanics for local realism.
\subsection{The set-up of experiment}\hfill \break
Consider maximally entangled state of two qubit given by the following Bell state
\begin{align*}
\ket{\Psi_{AB}}=\frac{1}{\sqrt{2}}(\ket{00}+\ket{11}.
\end{align*}
This state is shared by two parties space-like separated Alice and Bob. Alice has the first qubit and Bob the second qubit.
We also consider  three sets of orthogonal bases $\bf{X}, \bf{Y} \text{ and } \bf{Z}$,
\begin{align}
 \bf{X}&=\{\ket{x},\ket{X}\},\\
 \bf{Y}&=\{\ket{y},\ket{Y}\},\\
  \bf{Z}&=\{\ket{z},\ket{Z}\}.
\end{align}
Alice and Bob perform an arbitrary measurement on these sets of basis. If Alice  performs measurement on the bases $\bf{X}$  with result $\ket{x}$, then Bob's state collapses to $\ket{x}$. Similarly for other bases. Now, we will consider a scenario when a measurement on Alice part of the system is done before Bob part for two of these bases, namely, $\bf{X} \text{ and } \bf{Y} $. Then the state of Bob is steered to these bases and Bob  can performs measurement on his part for any of those three bases after his state collapse to the state.
Let's say the measurements Bob for any bases is,
\begin{align*}
\alpha &=|\bra{x}\ket{y}|^2,\\
\beta &=|\bra{x}\ket{z}|^2,\\
\gamma& =|\bra{y}\ket{z}|^2.
\end{align*}
The table below shows all possible outcomes according to quantum mechanics perdition  on Bob state, when Alice measures on  two of those bases $\bf{X} \text{ and } \bf{Y} $



\begin {table}[H]
\begin{center}\label{tabal1}
\begin{tabular}{ |c | c| c| c|c|}    
\hline
\head{A}&\head{B}&&\head{ basis Bob measures}\\
\hline
$\ket{x}$&$\Pr[\text{ Bob  measures }\ket{x}] = 1$&$\Pr[\text{ Bob  measures }\ket{X}] = 0$& $\bf{X}$\\
\hline
&$\Pr[\text{ Bob  measures } \ket{y}] = \alpha$&$\Pr[\text{ Bob  measures } \ket{Y}]= 1-\alpha$&$\bf{Y}$\\
\hline
&$\Pr[\text{ Bob  measures }\ket{z}] = \beta$&$\Pr[\text{ Bob  measures }\ket{Z}] =  1-\beta$&$\bf{Z}$\\
\hline
$\ket{X}$&$\Pr[\text{ Bob  measures }\ket{x}] = 0$&$\Pr[\text{ Bob  measures }\ket{X}]= 1$&$\bf{X}$\\
\hline
&$\Pr[\text{ Bob  measures }\ket{y}] =1-\alpha$&$\Pr[\text{ Bob  measures }\ket{Y}] = \alpha$&$\bf{Y}$\\
\hline
&$\Pr[\text{ Bob  measures }\ket{z}] = 1-\beta $&$\Pr[\text{ Bob  measures }\ket{Z}] = \beta$&$\bf{Z}$\\
\hline
$\ket{y}$&$\Pr[\text{ Bob  measures }\ket{y}] = 1$&$\Pr[\text{ Bob  measures }\ket{Y}] = 0$&$\bf{Y}$\\
\hline
&$\Pr[\text{ Bob  measure s }\ket{x}] = \alpha$&$ \Pr[\text{ Bob  measures }\ket{X}]= 1-\alpha$&$\bf{X}$\\
\hline
&$\Pr[\text{ Bob  measures }\ket{z}] = \gamma$&$\Pr[\text{ Bob  measures }\ket{Z}]= 1-\gamma$&$\bf{Z}$\\
\hline
$\ket{Y}$&$\Pr[\text{ Bob  measures }\ket{Y}] = 1$&$\Pr[\text{ Bob  measures }\ket{y}]= 0$&$\bf{Y}$\\
\hline
&$\Pr[\text{ Bob  measures }\ket{x}] = 1-\alpha$&$\Pr[\text{ Bob  measures }\ket{X}] = \alpha$&$\bf{X}$\\
\hline
&$\Pr[\text{ Bob  measures }\ket{z}]  = 1-\gamma$&$\Pr[\text{ Bob  measures }\ket{Z}] = \gamma$&$\bf{Z}$\\
\hline
\end{tabular}
\caption {Quantum prediction for Bob measurement when Alice measured on arbitrary basis from $\bf{X} \text{ and } \bf{Z}$. 
Where $\alpha, \beta \text{ and } \gamma \in [0,1]$.}
\label{table:3}
\end{center}
\end{table}





\subsection{ Construction of LHV model for two ensembles of $B$}\label{Triv}\hfill \break
Now we consider the case when system $B$ steered to two ensembles of orthogonal bases $\bf{X},\bf{y}$, by two different measurements on system $A$, where $\ket{x},\ket{X},\ket{y},\ket{Y}$ all different state for $B$.  From quantum mechanics of point view the density matrix for  the system $B$ on this ensembles is
\begin{equation}\label{STTS}
\rho_B=\frac{1}{2}\ket{x}\bra{x}+\frac{1}{2}\ket{X}\bra{X}=\frac{1}{2}\ket{y}\bra{y}+\frac{1}.{2}\ket{Y}\bra{Y}
\end{equation}
Let $v(\lambda)$ denote the ensembles of real state for $B$, then there must be a way to rearrange $v(\lambda)$ in the following form
\begin{equation}\label{Hidden}
v(\lambda)=\frac{1}{2}x(\lambda)+\frac{1}{2}X(\lambda)=\frac{1}{2}y(\lambda)+\frac{1}{2}Y(\lambda).
\end{equation}
From quantum states orthogonality of $\ket{x}, \ket{X}$ and  $\ket{y} ,\ket{Y}$, it must be possible to find a probability density over the space of real state $\lambda$ that can be decomposed into probability densities of $x(\lambda)$ and $X(\lambda)$, which have disjoint supports,
\begin{align*}
S_x\cap S_X=S_y\cap S_Y=\emptyset
\end{align*}

We can define four disjoint regions of real state $\lambda$ for the state system $B$:
\begin{align*}
S_1:=&S_x\cap S_y,\\
S_2:=&S_x\cap S_Y,\\
S_3:=&S_X\cap S_y,\\
S_4:=&S_X\cap S_Y.
\end{align*}


Now we consider measurements on state of Bob for the bases $\{\ket{x},\ket{X}\}$,$\{\ket{y},\ket{Y}\}$. To justify the principles of completeness and local realism as in  \citep{EPR} or at least the local realism, we must have set of real state that obey quantum result when the measurements on these bases are done.

For example if the measurement is done on the basis $\{\ket{x},\ket{X}\}$ after the state steered to the assembles $\ket{y}$, we are supposed to get,
\begin{equation}
\int_{S_x} d\lambda y(\lambda)=|\bra{x}\ket{y}|^2:=\alpha=\Pr[ Bob \text{ measures } \ket{x}]
\end{equation}
 Using  the four disjoin regions, we can define the probability of obtaining a result of measurement in one ensembles after the state steered to other ensemble as,
\begin{equation}
\eta_j=\int_{S_j} d\lambda \eta(\lambda),
\end{equation}
where $j=1,2,3,4$, and $\eta:= x,X,y \text{ and } Y$.

For any local realistic theorem we must obtain the following result,
\begin{align*}
x_1=&\int_{S_1} d\lambda y(\lambda)=\int_{S_x} d\lambda y(\lambda)=\alpha\\
x_2=&\int_{S_2} d\lambda x(\lambda)=\int_{ S_Y} d\lambda x(\lambda)=1-\alpha\\
X_3=&\int_{S_3} d\lambda y(\lambda)=\int_{S_X } d\lambda y(\lambda)=1-\alpha\\
X_4=&\int_{S_4} d\lambda Y(\lambda)=\int_{S_X} d\lambda Y(\lambda)=\alpha\\
y_1=&\int_{S_1} d\lambda x(\lambda)=\int_{ S_y} d\lambda x(\lambda)=\alpha\\
Y_2=&\int_{S_2} d\lambda x(\lambda)=\int_{ S_Y} d\lambda x(\lambda)=1-\alpha\\
y_3=&\int_{S_3} d\lambda X(\lambda)=\int_{S_y} d\lambda X(\lambda)=1-\alpha\\
Y_4=&\int_{S_4} d\lambda Y(\lambda)=\int_{S_X} d\lambda Y(\lambda)=\alpha,
\end{align*}
where $0\leq \eta_i\leq 1, \text{ for } i=1,2,3,4 \text{ and } \eta= x,X,y,Y$, and zero or one for the reset of $\eta_i$.

By substituting this result in local hidden variables equation in \ref{Hidden}, we get
\begin{align*}
v_1=v_4=\frac{\alpha}{2},\quad v_3=v_2=\frac{1-\alpha}{2}.
\end{align*}






\subsection{Constructionof LHV model for three ensembles}\label{ss3}\hfill \break
Suppose, Bob has measured to the third ensemble of orthogonal bases $\{\ket{z},\ket{Z}\}$, Then the density matrix for system $B$ would be in the following form,
\begin{equation}
\rho_B=\frac{1}{2}\ket{x}\bra{x}+\frac{1}{2}\ket{X}\bra{X}=\frac{1}{2}\ket{y}\bra{y}+\frac{1}{2}\ket{Y}\bra{Y}=\frac{1}{2}\ket{z}\bra{z}+\frac{1}{2}\ket{Z}\bra{Z}.
\end{equation}
In local hidden variable theorem, the corresponding probability distribution over set of real states $\lambda$ should has the following form,
\begin{equation}
v(\lambda)=\frac{1}{2} x(\lambda)+\frac{1}{2} X(\lambda)=\frac{1}{2} y(\lambda)+\frac{1}{2}Y(\lambda)=\frac{1}{2}z(\lambda)+\frac{1}{2}Z(\lambda),
\end{equation}
where $\delta(\lambda)$,$\delta=x,X,y,Y,z \text{ and } Z$ denote real state corresponding to quantum state$\ket{\delta}$.

Recall that,
\begin{align*}
|\bra{x}\ket{z}|^2&=\beta,\\
|\bra{y}\ket{z}|^2&=\gamma,\\
\end{align*}

In any local realism description of quantum reality, it must that all the following equations satisfied, when Bob measured on the basis $\bf{Z}$,
\begin{align}
z_1+z_2&=\int_{S_1} d\lambda z(\lambda)+\int_{S_2} d\lambda z(\lambda)=\int_{S_x} d\lambda z(\lambda)=\beta\label{Zies1}\\
z_1+z_3&=\int_{S_1} d\lambda z(\lambda)+\int_{S_3} d\lambda z(\lambda)=\int_{S_y} d\lambda z(\lambda)=\gamma\label{Zies2}\\
z_2+z_4&=\int_{S_2} d\lambda z(\lambda)+\int_{S_4} d\lambda z(\lambda)=\int_{S_Y} d\lambda z(\lambda)=1-\gamma\label{Zies4}\\
z_3+z_4&=\int_{S_3} d\lambda z(\lambda)+\int_{S_4} d\lambda z(\lambda)=\int_{S_X} d\lambda z(\lambda)=1-\beta\label{Zies3}\\
Z_1+Z_2&=\int_{S_1} d\lambda Z(\lambda)+\int_{S_2} d\lambda Z(\lambda)=\int_{S_x} d\lambda Z(\lambda)=1-\beta\label{Zies7}\\
Z_1+Z_3&=\int_{S_1} d\lambda Z(\lambda)+\int_{S_3} d\lambda Z(\lambda)=\int_{S_y} d\lambda Z(\lambda)=1-\gamma\label{Zies8}\\
Z_2+Z_4&=\int_{S_2} d\lambda Z(\lambda)+\int_{S_4} d\lambda Z(\lambda)=\int_{SY} d\lambda Z(\lambda)=\gamma\label{Zies6}\\
Z_3+Z_4&=\int_{S_3} d\lambda Z(\lambda)+\int_{S_4} d\lambda Z(\lambda)=\int_{S_X} d\lambda Z(\lambda)=\beta\label{Zies5}\\
v_1&=\frac{1}{2}z_1+\frac{1}{2}Z_1=\frac{\alpha}{2}\label{vies1}\\
v_2&=\frac{1}{2}z_2+\frac{1}{2}Z_2=\frac{1-\alpha}{2}\label{vies3}\\
 v_3&=\frac{1}{2}z_3+\frac{1}{2}Z_3=\frac{1-\alpha}{2}\label{vies4}\\
 v_4&=\frac{1}{2}z_4+\frac{1}{2}Z_4=\frac{\alpha}{2}\label{vies2}.
\end{align}
Since, $z_i\text{ and } Z_i$ for $i=1,2,3,4$  are integral over a probability density of real space, it must be away to represent $z_i, Z_i \in [0,1]$.

For any local realism interpretation of quantum theory, all these consistency condition above must be satisfied simultaneously for all physical that could present by $\alpha, \beta \text{ and } \gamma \in [0,1]$. This do not happen for all ensembles of state $B$, this contradiction implies that the quantum mechanics violates  local realism.
\subsection{Construction of LHV model gives QM  result for three ensembles}\hfill \break
In this section we present a proof for Conjecture 1  in \citet*{Jevtic:2015:10.1364/JOSAB.32.000A50} by measuring the state of the system $B$ for three assemblies of orthogonal basis.
Jevtic and Rudolph  show that if $\bf{Z}$ is orthogonal basis such that $\ket{z}$ bisects $\ket{x} \text{ and } \ket{y}$, then there is no local hidden variable model explain the QM results. For general case of $\alpha, \beta \text{ and } \gamma $ they gave the following Conjecture,



\begin{theorem}
The triple of overlaps $\alpha,\beta, \gamma$ demonstrates a violation of local realism if and only if the point $[\alpha,\beta, \gamma]$ does not lie in the convex hull of the four points
$[1, 0, 0], [0, 1, 0], [0, 0, 1], [1, 1, 1]$.
\end{theorem}
In the following, we present the proof of this Conjecture.
\begin{proof}
We will divided the proof into two parts, In the first part, we drive general solution for $z_i$ in terms of $[\alpha, \beta, \gamma]$ such that consistency conditions for LHV theory description gives a result same as quantum mechanics description using all possible analytical cases for the values of $\alpha,\beta $.


In the second part, we will do the same for Convex hull of the points $[1, 0, 0], [0, 1, 0], [0, 0, 1], [1, 1, 1]$  to show that we obtain the same solutions as a part one.

Part 1:

From equations \ref{vies1}, \ref{vies2}, \ref{vies3} and \ref{vies4}, we can write $Z_i$ in terms of $z_i$ and $\alpha$ as below.
\begin{align}
Z_1=&\alpha-z_1,\label{nik1}\\
Z_3=&1-\alpha-z_3,\label{nik3}\\
Z_2=&1-\alpha-z_2,\label{nik4}\\
Z_4=&\alpha-z_4,\label{nik2}.
\end{align}
By substituting the value of $Z_i$ in Equations from \ref{nik1} to \ref{nik4} into Equations \ref{Zies5} to \ref{Zies8} we see that these equations repetition for equations \ref{Zies1} to \ref{Zies4}. This reduce our consistency condition to,
\begin{align}
z_1+z_2&=\beta,\label{z1}\\
z_1+z_3&=\gamma,\label{z2}\\
z_2+z_4&=1-\gamma,\label{z4}\\
z_3+z_4&=1-\beta,\label{z3}\\
0\leq z_1,z_4& \leq \alpha,\label{z5}\\
0\leq z_2,z_3 &\leq 1- \alpha.\label{z6},
\end{align}
but solving the system of equations from \ref{z1} to \ref{z6} is equivalent to solving the following system, because the second equation is the first plus the fourth minus the third,
\begin{align}
z_1+z_2&=\beta,\label{z1}\\
z_3+z_4&=1-\beta,\label{z32}\\
z_2+z_4&=1-\gamma,\label{z42}\\
0\leq z_1,z_4& \leq \alpha,\label{z52}\\
0\leq z_2,z_3 &\leq 1- \alpha.\label{z62},
\end{align}
To solve this system of linear equations, we will use case analysis  $\beta, \alpha  \in [0,1]$, then we have four  possible cases.

\begin{itemize}
\item Case 1 . $1-\beta \leq 1-\alpha , \alpha \leq \beta$

Under this possibility, finding solution for the system of equations form  \ref{z1} to \ref{z62} is equivalent to.
\begin{align*}
\beta -\alpha&\leq z_2 \leq 1-\alpha\\
0 &\leq z_4 \leq 1-\beta\\
z_2&+z_4=1-\gamma,
\end{align*}
which has a general solution in terms of $\alpha,\beta \text{ and } \gamma$,
\begin{equation}\label{zc1}
 \beta -\alpha\leq1-\gamma \leq 2-\beta-\alpha.
\end{equation}


\item Case 2. $\beta \leq \alpha , 1-\alpha \leq 1-\beta$

Under these conditions the solution of system inequalities take the following form,
\begin{align*}
0&\leq z_2\leq \beta\\
\alpha-\beta&\leq z_4\leq \alpha\\
z_2&+z_4=1-\gamma,
\end{align*}
Which can be rewritten in the following form,
\begin{equation}\label{zc3}
\alpha-\beta \leq 1-\gamma \leq \alpha+\beta.
\end{equation}

\item Case 3. $\alpha\leq \beta , 1-\beta \leq 1-\alpha$.

Under these cases finding solution for system of equations from \ref{z1} to \ref{z62}, equivalent to solving the following inequality.
\begin{align*}
\beta-\alpha \leq &z_2 \leq \beta\\
0 \leq &z_4 \leq \alpha\\
z_2+z_4&=1-\gamma
\end{align*}
which has  general solution in the following form
\begin{equation}\label{zc5}
\beta-\alpha \leq1-\gamma \leq \beta+\alpha.
\end{equation}
\item Case 4. $1-\alpha\leq  \beta , 1-\beta \leq \alpha$:

in this cases the solution of our system equivalent to solution of the system below.
\begin{align*}
0\leq z_2&\leq 1-\alpha\\
\alpha-\beta &\leq z_4\leq 1-\beta\\
z_2+z_4&=1-\gamma
\end{align*}
then the system solution take the following form
\begin{equation}\label{zc7}
\alpha-\beta \leq 1-\gamma \leq 2-\alpha-\beta.
\end{equation}
\end{itemize}

Part 2:

Points $[\alpha,\beta,\gamma]$ in the convex hull of the points $[1, 0, 0], [0, 1, 0], [0, 0, 1], [1, 1, 1]$ can be described the following set of linear inequalities.
\begin{align}
\alpha+\beta+\gamma &\geq 1,\label{con1}\\
\alpha+\beta-\gamma &\leq 1,\label{con2}\\
\alpha-\beta+\gamma &\leq 1,\label{con3}\\
-\alpha+\beta+\gamma &\leq 1\label{con4},
\end{align}
for all, $ \alpha, \beta \text{ and }\gamma \in [0,1]$.

By applying the same analytical cases as in Part 1, we will solve this system of linear  inequalities above in an attempts to drive the general solution of this system.
\begin{itemize}
\item Case 1 . $1-\beta \leq 1-\alpha,\alpha \leq \beta$.

The inequalities \ref{con1} and \ref{con2} are always satisfied int his case, so from equations \ref{con3} and \ref{con4} we have
\begin{equation}
\beta-\alpha \leq 1-\gamma \leq 2-\beta-\alpha, 
\end{equation}
but this is same as what get Part 1  inequality \ref{zc1}.
\item Case 2.$\beta \leq \alpha,1-\alpha \leq 1-\beta$

For this case the inequalities \ref{con4} and \ref{con2} are always satisfied conditions. We have the following result from \ref{con1} and \ref{con3} 
\begin{equation}\label{chc3}
\alpha-\beta \leq 1-\gamma \leq \alpha+\beta.
\end{equation}
The result in \ref{chc3} is same as in \ref{zc3}.

\item  Case 3. $\alpha\leq \beta,\ 1-\beta \leq 1-\alpha$.

For this case  \ref{con3} and \ref{con2} extra constraints, this implies the result below.
\begin{equation}
\beta -\alpha \leq 1-\gamma\leq \beta +\alpha
\end{equation}
since, this what we have in \ref{zc5}.
\item  Case 4  $1-\alpha \leq 1-\beta,\beta  \leq \alpha$.

The condition in \ref{con1} and \ref{con2} are  always satisfied in this case, so from \ref{con3} and \ref{con4}, we have
\begin{equation}
\beta-\alpha \leq 1-\gamma \leq 2-\beta-\alpha, 
\end{equation}
This result the exact result in \ref{zc7} .
\end{itemize}
Hence, the general solution for the local hidden variables equations  and convex hull inequalities of given points are the same for  all possible arrangement of $\alpha, \beta , \gamma \in [0,1]$.
\end{proof}