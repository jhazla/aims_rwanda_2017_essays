\chapter{Introduction}
%\Jnote{I think thesis title should be:
%  ``Different Approaches to Quantum Nonlocality''.}
%
%\Jnote{This is way too short, especially the part summarizing your work.}

In a macroscopic description of nature, the deterministic theories of physics offer good predictions for the dynamics of any physical system. All measured quantities agree with the principle of locality and there is no doubt about the reality associated with physical state. According to annotation form special relativity distant objects cannot have direct influence on each other. That is, an object is directly influenced only by its immediate surroundings.

On the other hand, the remarkable feature of the microscopic world prescribed by quantum mechanics is the statistical nature of a result obtained from the state of system subject to measurement. Unlike the classical statistical mechanics, the superposition and interferences in quantum mechanics cause what is known as the measurement problem and  quantum non-locality.
The process of the measurement itself is subject to philosophical debate.
%\Jnote{Statistical mechanics is not deterministic, same as QM. The differences
%are in interference (which causes measurement problem) and nonlocality of QM.}

In 1935, Einstein, Podolsky and Rosen in the seminal paper \citep*{EPR}, claimed for a certain quantum state, two non commutative  operators corresponding to two physical quantities can be measured simultaneously.
This seems to contradict the Heisenberg uncertainty principle, which says that a pair of non commutative operators corresponding to two physical quantities  can not be measured simultaneously, in other words, no measurements on certain system can be made without affecting the system state. They also doubted the completeness of the reality associated with quantum mechanics and they suggested local hidden variables theory as a completion to the quantum wave function description of a physical reality.
%\Jnote{One para on EPR, one para on Schroedinger.}
 %\Jnote{Don't copy sentences from Internet.}
They drew a conclusion that either the physical reality that described by this theory (quantum mechanics) is not complete or the theory itself is not complete, because of locality violation. %According to the experimental results from  a macroscopic system can not be happening. 
A year latter, Schrödinger introduced the concept of quantum steering in an attempt to formalize the EPR paradox \citep*{schrodinger1935discussion}.

In 1964, John S. Bell showed that the principle of local causes is incompatible with quantum mechanics and there is no local hidden variables theory can give the quantum predictions \citep*{book:800289}. An experimentally testable form of Bell's theorem was proposed by John Clauser, Michael Horne, Abner Shimony and Richard Holt \citep*{PhysRevLett.23.880}. The experimental results also become in supports of quantum mechanics predictions \citep{rowe2001experimental,PhysRevLett.28.938,PhysRevLett.49.1804,PhysRevLett.81.5039}.

Meanwhile, in the  modern quantum information theory, the quantum  violation of Bell's theorem is considered in the form of multi-prover games. The classical games are modules study by following particular model. In the game, there are two or more players called Alice, Bob and so on. There is also a referee who communicates with the players by asking them questions and receiving  the answers, then decides whether the players win the game or not. In addition the plays are not allowed to communicate at all during the game time \citep*{CHALLET1997407}. In the quantum analogue, the players are consider to be shared an entangled state of two qubits or more  and they perform different quantum operators.

In addition to their own interest, the quantum games offer a new in sight for fascinating world of quantum information  \citep*{PhysRevLett.82.1052} and this is used in the analysis the quantum advantages in information technology sectors such as quantum cryptography, communication and teleportation.
%%%%\Jnote{CHSH proposed an experiment. Try finding references for some experimental papers
%%%%that implement the experiment.}
%%%%
%%%%\Jnote{Write that modern quantum information theory models violations
%%%%  of Bell's theorem using two-prover games with entanglement.
%%%%  Cite some examples, you can find them in Section 3 of
%%%%  https://arxiv.org/pdf/quant-ph/0404076.pdf
%%%%  and in Watrous's lecture (GHZ game).}
%he field of quantum games is currently attracting much attention within the physics community. In addition to their own intrinsic interest%, 


In this essay, we approached quantum non-locality from two points of view. In the first section of chapter three, we investigate quantum  non-locality  impact on the famous CHSH ($XOR$) game.  Basically, We show  that the quantum strategy for this game improved the winning probability. In section two, we present the proof of T'sirelson's bound \citep*{Cirel'son1980} to show that quantum strategy in the first part is optimal. It is interesting to see that an entanglement gives only limited advantages for this game compared to classical strategy.

Although, we show that there exist a non-trivial game for which no advantage can be gained from quantum entanglement. 
The $NAND$ game is also binary game like $CHSH$ game with only difference in the winning condition. In it the player win if and only if  $NAND$ of their answers equal to $AND$ of the asked equations. We show that for such game the classical optimal strategy is same as the quantum. We do this using similar approach to T'sirelson's upper bound.



%
%\Jnote{Separate paragraphs: first, say that in the first part of Chapter two
%  you show quantum strategy for CHSH game.}
%
%\Jnote{Then, you present the proof of Tsirelson's bound showing that
%  quantum strategy from first part is optimal. Motivate it by saying that
%  it is interesting to see that entanglement gives you only limited
%  advantage.}
%
%\Jnote{Continue saying that there exist non-trivial games for which
%  there is no quantum advantage and that you give an example of such
%  game. Explain relation to XOR game.}

In Chapter four, we used more fundamental aspects to deal with the inconsistency of local hidden variables theory and with the statistical predictions of quantum mechanics.
This is built on more basic way to reformulate Bell's theory given by  Jevtic and Rodulph \citep*{Jevtic:2015:10.1364/JOSAB.32.000A50}. They used quantum steering  in an attempt to derive  quantum mechanics  violation of local realism in the form of hidden variables. 
In particular, they showed when a subsystem of two qubit entangled state is steered to three ensembles of orthogonal  bases and the third  ensemble of those orthogonal bases bisect the first two, there is no local hidden variables theory could  provide quantum result.

Jevtic and Rodulph  provided a conjecture for the general case. That is, when a subsystem of maximally entangled state of two qubits is steered to three ensemble of orthogonal bases, then quantum mechanics violate the local realism in the form hidden variables if and  only if the angles between these bases do not lie in convex hull of certain points. 

We investigate their model more and also provided the proof for the conjecture using case analysis for different configuration of angles. 

%\Jnote{More paragraphs. Explain (high-level) what JR is about.
%  Explain (high-level) what the conjecture states and how we solve it.
%  This should be at least four paragraphs.}