\chapter{Introduction}
In a macroscopic description of the nature, the deterministic theories of physic offer good prediction for dynamics of any physical system.
All measured quantities a commonly agreed with principle of locality and there is no doubts about the reality associated with physical state. Therefore, no theoretical prediction or experimental result violate this. The idea that distant objects cannot have direct influence on one another.That is, an object is directly influenced only by its immediate surroundings environment.

On the other hand, the remarkable features of the microscopic world prescribed by quantum theory is the statistical restriction of a result obtained from the state of system subject to measurement. Unlike classical statistical mechanics,the quantum mechanic result on the average is not deterministic.
The process of measurement itself subject to philosophical debate, which lead to discovery of Quantum myths.

In 1935,Einstein, Podolsky and Rosen in well known paper claim that quantum  mechanics is not complete theory \citep*{EPR}.
A year later, Schrödinger introduced the concept of quantum steering in an attempt to formalize the EPR paradox \citep*{schrodinger1935discussion},which in it Einstein, Podolsky and Rosen doubts about the completeness and the reality of the quantum mechanics theory, and they suggested local hidden variable theorem as completion of quantum wave function description of a physical reality \citep*{EPR}.
 
The idea of non-locality, what Albert Einstein rather dismissively called “spooky actions at a distance. This was first described in the “EPR papers”\citep*{EPR}, which they concluded by either the physical reality that described by this theorem (quantum mechanics) not complete or theorem itself not complete, because of locality violation.This according to the experimental results from  macroscopic system can not be happening. 

In 1964,John S. Bell showed that, the principle of local causes is incompatible with quantum mechanics and there is no local hidden variables theory can preserve quantum locality \citep*{book:800289}. An experimental verification for Bell's theory provided by John Clauser, Michael Horne, Abner Shimony, and Richard Holt \cite*{PhysRevLett.23.880}.

In this essay, we approached quantum non-locality from two points of views.In chapter two, we investigated the non locality  impact for binary games strategy. When two  players share entangled state of two qubits and performs different quantum operators. basically, We show quantum strategy for two binary games ($XOR$ and $NAND$). In the first game the non-local strategy has increased the winning probability  compare to the classical strategy, with upper bound given by Tsirelson inequality\citep*{Cirel'son1980}. In the second game no notable benefit  gained from quantum non-locality.

In the second point, which covered in chapter three. We deal with quantum non-locality  in more fundamental aspect. The inconsistency of local hidden variables theory with the statistical predictions of quantum mechanics.
This built on more basic way to reformulate Bell's theory given by  Jevtic and Rodulph \citep*{Jevtic:2015:10.1364/JOSAB.32.000A50}. Also, we  proof conjecture 1  from this paper. 



