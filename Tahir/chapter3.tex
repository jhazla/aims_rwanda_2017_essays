\chapter{Bell's Theorem via Steering theorem}
Our motivation built up on Jevtic and  Rodulph work in 2015 \citep{Jevtic:2015:10.1364/JOSAB.32.000A50} , which in it They used a simple approach to show the violation of quantum mechanics for local realism in the form Of hidden variables theory.The impossibility of local hidden variables interpretation for quantum mechanics was illustrated in 1964 by Bell in his well celebrated theory \cite{bell1964einstein}.

\section{Introduction}\hfill \break
In 1935,Einstein, Podolsky and Rosen in well known paper claim that quantum  mechanics is Not complete theory \citep{EPR}.
A year later, Schrödinger introduced the concept of quantum steering in an attempt to formalize the EPR paradox \citep{schrodinger1935discussion},which in it Einstein, Podolsky and Rosen are doubts about the completeness and the reality of the quantum mechanics theory, and they suggested local hidden variable theory as completion of quantum wave function description of a physical reality \citep{EPR}. Quantum steering refers to the fact that, in a bipartite scenario, any local measurements  done by one of the parties can change the state of the other distant party.

Consider two player space like separated, Alice and Bob, who share a bipartite quantum state $\ket{\Psi}$ with reduced states$\ket{\Psi_A}$ and$\ket{\Psi_B}$ for the two parties, respectively. Alice can perform a collection of local measurements from set of observables $\{M_A\}$ for her part, each local measurement by Alice steer Bob part to new state \citep{book:474706}.To illustrate more, consider Alice has two sets orthogonal basis of ensembles, namely $\{\ket{u_n}\}$ and $\{\ket{v_n}\}$, then Alice part of entangled state $\ket{\Psi}$ for this assembles of basis is given.
\begin{equation}
\ket{\Psi}=\sum_n c_n \ket{\phi} \ket{u_n}=\sum_n d_n \ket{\varphi}\ket{v_n}
\end{equation}
The fact that $\ket{\phi} \text{ and} \ket{\varphi}$ are different state for Bob part used by EPR in their argument for hidden variables theory, as will as in their claim about incompleteness of Quantum mechanics.

 In 2007,Wiseman, Jones and Doherty formalised steering in terms of the incompatibility of quantum mechanical predictions with a classical-quantum model where pre-determined states are sent to the parties. Furthermore, the observation of quantum steering can also be seen as the detection of entanglement as well as the ability of not acceptance local hidden variables description\cite{Jevtic:2015:10.1364/JOSAB.32.000A50}.


In general, in any hidden variable theory, each Qubit has an assigned value for each observable, determined by a hidden variable (or a set of hidden variables) of real
states $\lambda$. A statistical ensembles of particles has a certain distribution $x(\lambda)$ of the hidden variable and thus the average value of an observable $A$ is given by
\begin{equation}
v=\int d \lambda x(\lambda),
\end{equation}
where $x(\lambda)$ is the set all hidden variables assigned to the observable $A$.


\section{Experiment}\hfill \break
\begin{theorem}
Given an entanglement state $\ket{\Psi_{AB}}$ of two system $A$ and $B$ the measurement on system $A$ collapse the system $B$ to the ensemble of states$\{\ket{\Phi_i}\}$ with associated probability $p_i$.If and only if 
\begin{equation}
\rho_B=\sum_i p_i \ket{\Phi_i} \bra{\Phi_i},
\end{equation}
where $\rho_B$ is subsystem density matrix driven from density matrix of engagement system by taking a partial trace over  system $A$.
\end{theorem}

\subsection{The Set-up of Experiment}\hfill \break
Consider maximally entanglement  state of two Qubit given by the following Bell state
\begin{align*}
\ket{\Psi_{AB}}=\frac{1}{\sqrt{2}}(\ket{00}+\ket{11}.
\end{align*}
This state shared by two parties $A$ and $B$ space like separated, $A$ has the first qubit and $B$ the second qubit.We also consider  three sets of orthogonal basis $\bf{X}, \bf{Y} \text{ and } \bf{Z}$.
\begin{align}
 \bf{X}&=\{\ket{x},\ket{X}\}\\
 \bf{Y}&=\{\ket{y},\ket{Y}\}\\
  \bf{Z}&=\{\ket{z},\ket{Z}\}
\end{align}


Since the state steerable from both sides, we will consider a scenario when a measurement on the system $A$ done before $B$ for two of these sets, namely, $\bf{X} \text{ and} \bf{Y} $, then the state of $B$ be steered to these basis.If $B$ performs measurement on his part for any of this three sets after his state collapse to assemblies which on it $A$ performed her measurement.
The quantum mechanics perdition given in the following table.


\begin {table}[H]
\begin{center}\label{tabal1}
\begin{tabular}{ |c | c c| c|}    
\hline
A&B\\
\hline
$\ket{x}$&$\bra{x}\ket{x} = 1$&$\bra{X}\ket{x} = 0$\\
\hline
&$\bra{y}\ket{x} = \alpha$&$\bra{Y}\ket{x} = 1-\alpha$\\
\hline
&$\bra{z}\ket{x} = \beta$&$\bra{Z}\ket{x} =  1-\beta$\\
\hline
$\ket{X}$&$\bra{x}\ket{X} = 0$&$\bra{X}\ket{X} = 1$\\
\hline
&$\bra{y}\ket{X} =1-\alpha$&$\bra{Y}\ket{X} = \alpha$\\
\hline
&$\bra{z}\ket{X} = 1-\beta $&$\bra{Z}\ket{X} = \beta$\\
\hline
$\ket{y}$&$\bra{y}\ket{y} = 1$&$\bra{Y}\ket{y} = 0$\\
\hline
&$\bra{x}\ket{y} = \alpha$&$\bra{X}\ket{y} = 1-\alpha$\\
\hline
&$\bra{z}\ket{y} = \gamma$&$\bra{Z}\ket{y} = 1-\gamma$\\
\hline
$\ket{Y}$&$\bra{x}\ket{Y} = 1-\alpha$&$\bra{X}\ket{Y} = \alpha$\\
\hline
&$\bra{y}\ket{Y} = 0$&$\bra{Y}\ket{y} = 1$\\
\hline
&$\bra{z}\ket{Y}  = 1-\gamma$&$\bra{Z}\ket{Y} = \gamma$\\
\hline
\end{tabular}
\caption {Quantum prediction}
\label{table:3}
\end{center}
\end{table}
Where $\alpha, \beta \text{ and } \gamma \in [0,1]$.





\subsection{LHV Prediction for two assemblies of $B$}\label{Triv}\hfill \break
Now we consider the case when system $B$ steered to two ensembles of orthogonal basis$\{\ket{x},\ket{X}\},\{\ket{y},\ket{Y}\}$, by two different measurements on system $A$.
Where $\ket{x},\ket{X},\ket{y},\ket{Y}$ all different state for $B$,  from quantum mechanics of point view the density matrix for  the system $B$ on this ensembles
\begin{equation}\label{STTS}
\rho_B=\frac{1}{2}\ket{x}\bra{x}+\frac{1}{2}\ket{X}\bra{X}=\frac{1}{2}\ket{y}\bra{y}+\frac{1}{2}\ket{Y}\bra{Y}
\end{equation}
Let's $v(\lambda)$ donate ensembles of real state for $B$, then there must a way to rearrange $v(\lambda)$ in the following form.
\begin{equation}\label{Hidden}
v(\lambda)=\frac{1}{2}x(\lambda)+\frac{1}{2}X(\lambda)=\frac{1}{2}y(\lambda)+\frac{1}{2}Y(\lambda)
\end{equation}
From quantum states orthogonality of $\ket{x} \ket{X}$ and  $\ket{y} \ket{Y}$, it must be possible to find a probability density over some space of real state $\lambda$ such that we can be decomposed into probability densities of $x(\lambda)$ and $X(\lambda)$ which are disjoint.

We can define four disjoin region of real state $\lambda$ for the state system $B$  such that the orthogonality condition of quantum state for each pair justified.
\begin{align*}
S_1:=&S_x\cap S_y\\
S_2:=&S_x\cap S_Y\\
S_3:=&S_X\cap S_y\\
S_4:=&S_X\cap S_Y\\
\end{align*}


Now we consider a projector measurements on state $B$ for the basis $\{\ket{x},\ket{X}\}$,$\{\ket{y},\ket{Y}\}$  which all different states for system $B$. To  justify the principles of completeness and local realism as in  \citep{EPR} or at least the local reality principle, we must have set of real state obey quantum result when a projector measurement on these basis done.

For example if the measurement done on the basis $\{\ket{x},\ket{X}\}$ after the state steered to the assembles $\{\ket{y},\ket{Y}\}$ we suppose to get.
\begin{equation}
\int_{S_x} d\lambda y(\lambda)=|\bra{x}\ket{y}|^2:=\alpha
\end{equation}
Where $\alpha \in [0,1]$, using  the four disjoin regions, we can define the probability that obtaining a result of measurement in one ensembles after the state steered to anther ensembles as.
\begin{equation}
\eta_j=\int_{S_j} d\lambda \eta(\lambda)
\end{equation}
Where $j=1,2,3,4$,$\eta:= x,X,y \text{ and} Y$.

For any local realistic theory we must obtain the following result.
\begin{align*}
x_1=&\int_{S_1} d\lambda y(\lambda)=\alpha\\
x_2=&\int_{S_2} d\lambda x(\lambda)=1-\alpha\\
X_3=&\int_{S_3} d\lambda y(\lambda)=1-\alpha\\
X_4=&\int_{S_4} d\lambda Y(\lambda)=\alpha\\
y_1=&\int_{S_1} d\lambda x(\lambda)=\alpha\\
Y_2=&\int_{S_2} d\lambda x(\lambda)=1-\alpha\\
y_3=&\int_{S_3} d\lambda X(\lambda)=1-\alpha\\
Y_4=&\int_{S_4} d\lambda X(\lambda)=\alpha
\end{align*}
Where $0\leq \eta_i\leq 1, \text{ for} i=1,2,3,4 \text{ and} \eta= x,X,y,Y$, and zero or one for the reset of $\eta_i$.

 By substitution this result in local hidden variables equation in \ref{Hidden},the following result is obtain.
\begin{align*}
v_1=v_4=\frac{\alpha}{2}, v_3=v_2=\frac{1-\alpha}{2}
\end{align*}






\subsection{LHV Prediction for three ensembles}\hfill \break
In this section we consider a proof for a conjecture 1  in\citep{Jevtic:2015:10.1364/JOSAB.32.000A50} by steering the state of the system $B$ for three assemblies of orthogonal basis.To be more precise the only change we made for  conjecture 1 is we rename a theorem without any modification in the expression  \citep{Jevtic:2015:10.1364/JOSAB.32.000A50}.



Now suppose third measurement on $A$  steered the state of $B$ to third ensemble of orthogonal state$\{\ket{z}+\ket{Z}\}$, then the density matrix for system $B$ would be in the following form.
\begin{equation}
\rho_B=\frac{1}{2}\ket{x}\bra{x}+\frac{1}{2}\ket{X}\bra{X}=\frac{1}{2}\ket{y}\bra{y}+\frac{1}{2}\ket{Y}\bra{Y}=\frac{1}{2}\ket{z}\bra{z}+\frac{1}{2}\ket{Z}\bra{Z}
\end{equation}
In local hidden variable theorem the correspond probability distribution over set of real states $\lambda$ should has the following form.
\begin{equation}
v(\lambda)=\frac{1}{2} x(\lambda)+\frac{1}{2} X(\lambda)=\frac{1}{2} y(\lambda)+\frac{1}{2}Y(\lambda)=\frac{1}{2}z(\lambda)+\frac{1}{2}Z(\lambda).
\end{equation}
Where $\delta(\lambda)$,$\delta=x,X,y,Y,z \text{ and} Z$ donate real state corresponding to quantum state$\ket{\delta}$.

When a measurement  done on basis of $\{\ket{x},\ket{X}\}$ and/or$\{\ket{y},\ket{Y}\}$  after the state of system has  been $B$ steered to assemblies $\{\ket{z},\ket{Z}\}$, quantum mechanics prediction in table  \ref{table:3}.
\begin{align*}
|\bra{x}\ket{z}|^2&=\beta\\
|\bra{y}\ket{z}|^2&=\gamma\\
|\bra{X}\ket{z}|^2&=|\bra{x}\ket{Z}|^2=1-\beta\\
|\bra{y}\ket{Z}|^2&=|\bra{Y}\ket{z}|^2=1-\gamma
\end{align*}

In any local realism description of quantum reality, it must be away such that all the following condition satisfied, when projector a measurement done on the basis of first and second orthogonal ensembles of system $B$, after it state steered to assemblies $\{\ket{z},\ket{Z}\}$

\begin{align}
z_1+z_2&=\int_{S_1} d\lambda z(\lambda)+\int_{S_2} d\lambda z(\lambda)=\int_{S_x} d\lambda z(\lambda)=\beta\label{Zies1}\\
z_1+z_3&=\int_{S_1} d\lambda z(\lambda)+\int_{S_3} d\lambda z(\lambda)=\int_{S_y} d\lambda z(\lambda)=\gamma\label{Zies2}\\
z_3+z_4&=\int_{S_3} d\lambda z(\lambda)+\int_{S_4} d\lambda z(\lambda)=\int_{S_X} d\lambda z(\lambda)=1-\beta\label{Zies3}\\
z_2+z_4&=\int_{S_2} d\lambda z(\lambda)+\int_{S_4} d\lambda z(\lambda)=\int_{S_Y} d\lambda z(\lambda)=1-\gamma\label{Zies4}\\
Z_3+Z_4&=\int_{S_3} d\lambda Z(\lambda)+\int_{S_4} d\lambda Z(\lambda)=\int_{S_X} d\lambda Z(\lambda)=\beta\label{Zies5}\\
Z_2+Z_4&=\int_{S_2} d\lambda Z(\lambda)+\int_{S_4} d\lambda Z(\lambda)=\int_{SY} d\lambda Z(\lambda)=\gamma\label{Zies6}\\
Z_1+Z_2&=\int_{S_1} d\lambda Z(\lambda)+\int_{S_2} d\lambda Z(\lambda)=\int_{S_x} d\lambda Z(\lambda)=1-\beta\label{Zies7}\\
Z_1+Z_3&=\int_{S_1} d\lambda Z(\lambda)+\int_{S_3} d\lambda Z(\lambda)=\int_{S_y} d\lambda Z(\lambda)=1-\gamma\label{Zies8}\\
v_1&=\frac{1}{2}z_1+\frac{1}{2}Z_1=\frac{\alpha}{2}\label{vies1}\\
v_4&=\frac{1}{2}z_4+\frac{1}{2}Z_4=\frac{\alpha}{2}\label{vies2}\\
v_2&=\frac{1}{2}z_2+\frac{1}{2}Z_2=\frac{1-\alpha}{2}\label{vies3}\\
 v_3&=\frac{1}{2}z_3+\frac{1}{2}Z_3=\frac{1-\alpha}{2}\label{vies4}
\end{align}
Since, $z_i\text{ and } Z_i$ for $i=1,2,3,4$  integral over a probability density of real space, it must be away to represent them as non negative quantises.

For any local realism interpretation of quantum theory all these consistency condition above must be satisfy simultaneously for $\alpha, \beta \text{ and } \gamma \in [0,1]$.Which do not happen for all ensembles of state $B$, this obvious contradiction implies the Bell's theory.

\begin{theorem}
The triple of overlaps $\alpha,\beta, \gamma$ demonstrate a violation of local realism if and only if the point $[\alpha,\beta, \gamma]$ does not lie in the convex hull of the four points
$[1, 0, 0], [0, 1, 0], [0, 0, 1], [1, 1, 1]$.
\end{theorem}

\begin{proof}\hfill \break
We will divided the proof for two part,In the first part we drive general solution for $z_i$ in terms of $[\alpha, \beta, \gamma]$ such that consistency conditions for LHV theory description gives a result same as quantum mechanics description, using simple assumption for the values of $\alpha,\beta $.


In the second part, we will use the same simple assumptions for Convex hull of the points $[1, 0, 0], [0, 1, 0], [0, 0, 1], [1, 1, 1]$  to show how it flow our result in part one.

Part 1:

From equations \ref{vies1},\ref{vies2},\ref{vies3} and \ref{vies4} we can write $Z_i$ in terms of $z_i$ and $\alpha$ as below.
\begin{align}
Z_1=&\alpha-z_1\label{nik1}\\
Z_4=&\alpha-z_4\label{nik2}\\
Z_3=&1-\alpha-z_3\label{nik3}\\
Z_2=&1-\alpha-z_2\label{nik4}
\end{align}
By substituting the value of $Z_i$ in equations from \ref{nik1}to \ref{nik4} into equations \ref{Zies5} to \ref{Zies8} we see that these equations repetition for equations \ref{Zies1} to \ref{Zies4}, this reduce our consistency condition to.
\begin{align}
z_1+z_2&=\beta\label{z1}\\
z_1+z_3&=\gamma\label{z2}\\
z_2+z_4&=1-\gamma\label{z4}\\
z_3+z_4&=1-\beta\label{z3}\\
0\leq z_1,z_4& \leq \alpha\label{z5}\\
0\leq z_2,z_3 &\leq 1- \alpha\label{z6},
\end{align}
but solving the system of equations from \ref{z1} to \ref{z6} equivalent to solving the following system,
\begin{align}
z_1+z_2&=\beta\label{z1}\\
z_2+z_4&=1-\gamma\label{z42}\\
z_3+z_4&=1-\beta\label{z32}\\
0\leq z_1,z_4& \leq \alpha\label{z52}\\
0\leq z_2,z_3 &\leq 1- \alpha\label{z62},
\end{align}
To solve this system of linear equations we will use analytical approach, which in it we use quantum theory prediction $\beta, \alpha \text{ and } \gamma$ has to be have value in range $[0,1]$.
Since the solution is trivial when $\beta=\alpha=\gamma$, as  we demonstrated in section \ref{Triv}, then we have eight  possible cases.

\begin{itemize}
\item[• Case 1.] $1-\beta \leq 1-\alpha \leq \alpha \leq \beta$

Under this possibility  solution of the system of equations form  \ref{z1} to\ref{z62} equivalent to the solution of the following system of equations.
\begin{align*}
\beta -\alpha\leq &z_2 \leq 1-\alpha\\
0 \leq &z_4 \leq 1-\beta\\
z_2+z_4&=1-\gamma
\end{align*}
which has a general solution in terms of $\alpha,\beta \text{ and } \gamma$.
\begin{equation}\label{zc1}
 \beta -\alpha\leq1-\gamma \leq 2-\beta-\alpha
\end{equation}
for all $\alpha,\beta \in [0.5,1]$ and $, \gamma \in [0,1]$
\item[• Case 2.] $1-\beta \leq \alpha \leq 1-\alpha \leq \beta$

Under these assumption solving the system of equation form  \ref{z1} to\ref{z62} equivalent to 
\begin{align*}
\beta-\alpha \leq &z_2 \leq 1-\alpha\\
0 \leq &z_4 \leq 1-\beta\\
z_2+z_4&=1-\gamma
\end{align*}
which has a solution in form.
\begin{equation}\label{zc2}
\beta-\alpha \leq1-\gamma \leq 2-\alpha-\beta
\end{equation}
for $\alpha \in [0,0.5]$, $\beta\in [0.5,1]$ and $\gamma \in [0,1]$.
\item[• Case 3.] $\beta \leq \alpha \leq 1-\alpha \leq 1-\beta$

Under this condition solution for the system of equations take the following form
\begin{align*}
0&\leq z_2\leq \beta\\
\alpha-\beta&\leq z_4\leq \alpha\\
z_2&+z_4=1-\gamma.
\end{align*}
Which can be rewritten in the following form.
\begin{equation}\label{zc3}
\alpha-\beta \leq 1-\gamma \leq \alpha+\beta
\end{equation}
\item[• Case 4.] $\beta \leq 1-\alpha \leq \alpha \leq 1- \beta$
the system of equation has solution in form
\begin{align*}
0&\leq z_2 \leq\beta\\
\alpha-\beta&\leq z_4\leq \alpha\\
z_2&+z_4=1-\gamma
\end{align*}
In this the general solution is
\begin{equation}\label{zc4}
\alpha-\beta \leq 1-\gamma \leq \alpha+\beta
\end{equation}
\item[• Case 5.] $\alpha\leq \beta \leq 1-\beta \leq 1-\alpha$.

Under these assumption finding solution for system of equations from \ref{z1} to\ref{z62} equivalent to solving the following inequality.
\begin{align*}
\beta-\alpha \leq &z_2 \leq \beta\\
0 \leq &z_4 \leq \alpha\\
z_2+z_4&=1-\gamma
\end{align*}
which has solution in the following form
\begin{equation}\label{zc5}
\beta-\alpha \leq1-\gamma \leq \beta+\alpha
\end{equation}
for all $\alpha,\beta \in [0,0.5]$ and $\gamma \in [0,1]$.
\item[• Case 6.]$\alpha\leq  1-\beta \leq \beta \leq 1-\alpha$
In this case the system in has the flowing form
\begin{align*}
\beta - \alpha&\leq z_2 \leq \beta\\
0 \leq z_4 &\leq \alpha\\
z_2+z_4&=1-\gamma
\end{align*}
then the system solution take the following form
\begin{equation}\label{zc6}
\beta-\alpha \leq 1-\gamma \leq \alpha+\beta
\end{equation}
\item[• Case 7.] $1-\alpha\leq  \beta \leq 1-\beta \leq \alpha$:

in this case our system solution flow this form
\begin{align*}
0\leq z_2&\leq 1-\alpha\\
\alpha-\beta &\leq z_4\leq 1-\beta\\
z_2+z_4&=1-\gamma
\end{align*}
then the system solution take the following form
\begin{equation}\label{zc7}
\alpha-\beta \leq 1-\gamma \leq 2-\alpha-\beta
\end{equation}
\item[• Case 8.] $1-\alpha\leq 1- \beta \leq \beta \leq \alpha$:

Under this assumption solving the system equations from \ref{z1} to\ref{z62}, this equivalence to the following system of inequality.
\begin{align*}
0 \leq &z_2 \leq 1-\alpha\\
 \alpha-\beta& \leq z_4 \leq1 -\beta\\
z_2+z_4&=1-\gamma
\end{align*}
which has solution in the following form
\begin{equation}\label{zc8}
\alpha-\beta \leq1-\gamma \leq 2-\beta-\alpha
\end{equation}
for all $\alpha \in [0.5,1]$, $\beta \in [0,0.5]$ and  $\gamma \in [0,1]$.
\end{itemize}

Part 2:

Any points $[\alpha,\beta,\gamma]$ in convex hull of the points $[1, 0, 0], [0, 1, 0], [0, 0, 1], [1, 1, 1]$ can be satisfied the following set of linear equations.
\begin{align}
\alpha+\beta+\gamma &\geq 1\label{con1}\\
\alpha+\beta-\gamma &\leq 1\label{con2}\\
\alpha-\beta+\gamma &\leq 1\label{con3}\\
-\alpha+\beta+\gamma &\leq 1\label{con4},
\end{align}
for all, $ \alpha, \beta \text{ and }\gamma \in [0,1]$.

By applying same assumptions as in part 1 (case 1 to case 8), we will solve this system of linear  inequalities above in an attempts to drive the general solution of this system same as the one we have for $z_i$ in term of $ \alpha, \beta \text{ and }\gamma$.
\begin{itemize}
\item[• Case 1 and  2.] $1-\beta \leq 1-\alpha,\alpha ,1-\alpha \leq \beta$.

Equations \ref{con1} and \ref{con1} not necessary condition under those cases, so from equations \ref{con3} and \ref{con4} we have
\begin{equation}
\beta-\alpha \leq 1-\gamma \leq 2-\beta-\alpha, 
\end{equation}
but this same as what get for case 1 and case 2 in equations \ref{zc1} and \ref{zc2}
\item[• Case 3  and 4.] $\beta \leq \alpha,1-\alpha ,\alpha\leq 1-\beta$

For both case 3 and 4 the equations \ref{con4} and \ref{con2} are not necessary condition, so we have the following result from \ref{con1} and \ref{con3} 
\begin{equation}\label{chc3}
\alpha-\beta \leq 1-\gamma \leq \alpha+\beta.
\end{equation}
The result in \ref{chc3} is same as in \ref{zc3} and \ref{zc4}

\item[• Case 5 and 6.] $\alpha\leq \beta,\ 1-\beta ,\beta\leq 1-\alpha$.
For those case equations \ref{con3} and \ref{con2} extra constraints, this implies the result below.
\begin{equation}
\beta -\alpha \leq 1-\gamma\leq \beta +\alpha
\end{equation}
since the equation above it same as equations \ref{zc5} and \ref{zc6}
\item[• Case 7 and 8.] $1-\alpha \leq 1-\beta,\beta ,1-\beta \leq \alpha$.

Equation \ref{con1} and \ref{con1} are not necessary condition under those cases, so from equations \ref{con3} and \ref{con4} we have
\begin{equation}
\beta-\alpha \leq 1-\gamma \leq 2-\beta-\alpha, 
\end{equation}
This result the exact result in equations \ref{zc7} and \ref{zc8}.
\end{itemize}
Hence, the general solution for the local hidden variables equations  and convex hull of the given points gives  same result for  all possible arrangement of $\alpha, \beta , \gamma \in [0,1]$.
\end{proof}