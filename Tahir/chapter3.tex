\chapter{Bell's Theorem via Steering theorem}

Our  main goal in this chapter to drive Bell's non locality through quantum steering with a focus on two qubit bipartite  system and series of measurement using  different assemblies of bases.


\section{Steering Theory}

In 1935, as demonstration of EPR Schrödinger  introduced the concept of quantum steering in an attempt to formalise the EPR paradox \citep{schrodinger1935discussion},which in it Einstein, Podolsky and Rosen are doubt about the completeness and the reality of Quantum mechanics theory, and they suggested local hidden variable theory as completion of  quantum wave function description of  a physical reality \citep{EPR}. Quantum steering refers to the fact that, in a bipartite scenario,the local measurements  than done by one of the parties can change the state of the other distant party . In 2007,Wiseman, Jones and Doherty formalised steering in terms of the incompatibility of quantum mechanical predictions with a classical-quantum model where pre-determined states are sent to the parties. Furthermore, the observation of quantum steering can also be seen as the detection of entanglement as well as the ability of  accepting local hidden variable description\cite{Jevtic:2015:10.1364/JOSAB.32.000A50}.


Consider two parties space like separated, Alice and Bob, who share a bipartite quantum state $\ket{\Psi}$ with reduced states$\ket{\Psi_A}$ and$\ket{\Psi_B}$ for the two parties, respectively.  Alice can perform a collection of local measurements form set of observables $\{M_A\}$ for her part, each local measurement by Alice  steer Bob part to  new state  \cite{book:474706}.



\section{Local hidden variables}


\section{Steering Between Three Ensembles of Orthogonal State}
